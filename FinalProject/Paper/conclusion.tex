% -*- mode: LaTeX -*-
%%

%%%%%%%%%%%%%%%%%%%%%%%%%%%%%%%%%%%%%%%%%%%%%%%%%%%%%%%%%%%%%%%%%%%%%%%%%%%%%%%

\section{Conclusion}
\label{sec:conclusion}

In this paper we discussed radio frequency propagation. Specifically, we covered radio frequency propagation models, ran a few 
measurements on the POWDER platform, discussed our results, and compared the data between our propagation model and the 
ground truth. SPLAT! was our radio frequency propagation model of choice and its measurement results were compared with the 
Shout framework. Shout was developed by the POWDER team to conduct measurement studies on the POWDER platform. 

Our results showed that
SPLAT! on the 3561 MHz and 2620 MHz frequencies overpredicted the path loss, however on the 3550 MHz and 3690 MHz SPLAT!
under-predicted the path loss with respect to the Shout framework. Furthermore, on the 3561 MHz frequency, Shout and SPLAT! 
both picked the same two nodes with the best radio frequency propagation, but failed to choose the same two nodes when it came
to picking nodes with the worst radio frequency propagation. There is no definitive answer as to whether or not Shout has been
validated with respect to SPLAT!. Radio frequency propagation models take into account multiple parameters to calculate path loss,
it could just so be that SPLAT! is not the ideal model for the POWDER platform.

%%%%%%%%%%%%%%%%%%%%%%%%%%%%%%%%%%%%%%%%%%%%%%%%%%%%%%%%%%%%%%%%%%%%%%%%%%%%%%%
%% End of file.
