% -*- mode: LaTeX -*-
%%

%%%%%%%%%%%%%%%%%%%%%%%%%%%%%%%%%%%%%%%%%%%%%%%%%%%%%%%%%%%%%%%%%%%%%%%%%%%%%%%

\section{Introduction}
\label{sec:introduction}

With the creation and deployment of the Platform for Open Wireless Data-driven Experimental Research (POWDER)~\cite{Breen+:wintech20}
that started in early 2018 at the University of Utah, users have been able to work on wireless networking technologies to funnel their creative 
nature into innovative networking research experiments. One such experiment is Shout~\cite{shout}. Shout is a measurement framework 
developed by the POWDER team, that allows for radio frequency measurements to be conducted on the POWDER testbed. A measurement 
of particular interest involves the radio frequency propagation loss for each usable node on the platform. POWDER offers a wide range of 
wireless endpoints (i.e. rooftop nodes, static nodes at human height, nodes on campus shuttles and portable nodes) that can be used to 
measure radio frequency propagation. 

The goal of the work described in this paper is to use the POWDER platform to perform radio frequency measurements under different conditions and to compare
the results against radio frequency propagation tools. SPLAT!~\cite{splat} is a radio frequency signal propagation, loss, and terrain analysis
tool for the electromagnetic spectrum between 20 MHz and 20 GHz. For our purposes SPLAT! will serve as our modeling tool to predict
radio frequency propagation assuming the same scenarios. 

In this paper, we perform radio frequency propagation validation for the Shout framework using the modeling tool SPLAT! on the POWDER
platform. For the purposes of this paper we will be using Band 7 ($\sim$2600 MHz), Band 42 ($\sim$3500 MHz), and Band 43 ($\sim$3700 
MHz). We will be using the cbrssdr1 and cellsdr1 nodes as well as fixed endpoint nucs for our measurement collection. 

We perform three main activities. First, using Shout we will collect experimental data on the frequencies and nodes described above. 
Secondly, using SPLAT! we will predict radio frequency propagation assuming the same scenarios. Lastly, we will compare and analyze
any differences between the ``ground truth" measurements from Shout and the predictions obtained from SPLAT!. 

The contributions of this work are the following. We collect radio frequency propagation loss measurements on multiple bands for use 
in the POWDER platform. We create a SPLAT! research experiment profile that can be used for future radio frequency modeling within the 
POWDER platform. This includes the proper configuration files for every current node that is usable on the testbed. Lastly, we present our
results between the ``ground truth" and our simulated data.

%%%%%%%%%%%%%%%%%%%%%%%%%%%%%%%%%%%%%%%%%%%%%%%%%%%%%%%%%%%%%%%%%%%%%%%%%%%%%%%

%% End of file.