%% -*- mode: LaTeX -*-
%%

%%%%%%%%%%%%%%%%%%%%%%%%%%%%%%%%%%%%%%%%%%%%%%%%%%%%%%%%%%%%%%%%%%%%%%%%%%%%%%%

\begin{abstract}

Recently developed technologies and frameworks at the POWDER platform accelerate learning and innovate new forms of
networking research. On the other hand it is unclear how well these frameworks are performing and whether they meet their
evaluation standards. 

In this paper, we uncover the Shout framework. Shout is a framework developed by the POWDER team
to perform a variety of measurement studies on the platform. For our purposes we will use Shout to perform radio frequency 
propagation measurements on the POWDER nodes. To compare these results we will use an open source radio frequency
propagation modeling tool named SPLAT!. Due to the uncalibrated nature of the radios on POWDER we will use the path loss 
exponent as a form of assessment between the two. Furthermore, we will do a small terrain analysis of a chosen measurement
frequency to bring our results to the physical world and explain why we got the results we did. 
\\\\
Index Terms - RF propagation, Network Measurement, SPLAT!
\end{abstract}

%%%%%%%%%%%%%%%%%%%%%%%%%%%%%%%%%%%%%%%%%%%%%%%%%%%%%%%%%%%%%%%%%%%%%%%%%%%%%%%

%% End of file.
