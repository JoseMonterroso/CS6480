\documentclass[letterpaper,twocolumn,10pt]{article}
\usepackage{epsfig,xspace,url}
\usepackage{authblk}


\title{CS 6480: Paper reading summary\\
HA 10.a\\}
\author{José Monterroso}
\affil{School of Computing, University of Utah}

\begin{document}

\maketitle
\section{Title of paper: Making 5G a Reality}

Paper discussed in this summary is ``Making 5G a Reality''~\cite{5gwhite}.

\subsection{First pass information}
\label{sec:first}

\begin{enumerate}

\item {\it Category:}
This paper is a white paper relating to 5G technologies. Furthermore, homework assignment Ha9.a covered
the first half of this paper. For this homework assignment we will cover the second half of this paper. This 
paper is an analysis of the Third Generation Partnership Project (3GPP) 5G specifications. 

\item {\it Context:}
The technical area of this paper relates to 5G specifications and the impact on Business Support Systems 
(BSS) and Operational Support Systems (OSS). Furthermore, in the second half of this paper we cover the 
migration and interworking of 5G. As mentioned before this paper relates to most of our previously discussed 
papers. 5G will require the use of cloud computing and NFV to promote an agile and flexible network. 

\item {\it Assumptions:}  
During the reading of the second half of this paper there weren't any outstanding assumptions that the authors
made. The only overall assumption I noticed was that the authors assume the reader has prior knowledge of the 
4G infrastructure, this is because most if not all their 5G definitions relay on 4G equivalent technologies. 

\item {\it Contributions:}
The goal of this white paper still remains to bring information regarding 5G from 3GPP standardization
perspective and implications on BSS/OSS. It's paper contributes by bringing 5G information and implications
to CEOs and 5G product developers. 

\item {\it Clarity:} 
This paper from what I have read is not written poorly, but simply requires the reader to be well acquainted 
with 4G technologies, as well as computer networking jargon, and 5G abbreviations. 

\end{enumerate}

\subsection{Second pass information}
\label{sec:second}

\begin{itemize}

\item {\it Summary:} 
5G with 100\% coverage will not be available day one, so we need migration steps from 4G to 5G. The first step
for 5G is NSA, this means that 5G will work using the 4G core. Through the use of an upgraded EPC operators
can provide high data rate services and radio capacity to smart devices without waiting for the full 5G system to be 
rolled out. We then slightly touch on different methods to introduce 5G to the world, this highly depends on whether we have
initial investment into 5G technology for the area. Having different forms of migration means we need to consider different
forms of security. The primary idea is that when a UE connects to a 5G network the 5G security checks will be applied,
however in NSA with a combined network the UE will adhere to 4G security and will have limited access to 5G networks. Furthermore, 
security and access rights are checked when attaching to 4G or 5G networks, but also during handover between the two
networks. 5G will bring more choice to the user with regards to internet services, these services will be provided over any
type of network and services will become real-time and on-demand. Businesses will need to be ready day one to provide on 
demand services to different network slice customers. Similarly, the OSS environment will have to adapt to the impact of 5G. 
OSS will have to support delivery of complex services across different infrastructures. 

\end{itemize}

\subsection{Third pass information}
\label{sec:third}
\begin{itemize}

\item {\it Strengths:} 
The latter half of the paper had way less abbreviations and spent more of its time explaining the implications of 5G 
on existing businesses, and how those business should prepare themselves for the new 5G capabilities. I think the
figures in the latter half of the paper did a great job of visualizing end-to-end path using NFV technology, and the 
end-to-end orchestration architecture. However, I do wish that some of the other figures had better captions to 
explain what was going on in detail.  

\item {\it Weaknesses:} 
There were lots of cases throughout the latter half of the paper where bullet points were used. I think in order to make
the paper more readable the authors of the white paper should have divided the text into subsections or potentially 
have used a table to list some of the main bulleted points. Lastly, I wish they have spent more time on the details
surrounding the issues of end-to-end orchestration in 5G technologies using AI. I feel that orchestration has been 
such a major part of new internet technologies like cloud based services and NFV, so for them to not go in depth
on orchestration was a mistake. 

\item {\it Questions:} 
I've actually never have heard or have seen the use of the term 'omni-channel ecosystem' so I look forward to 
researching the term. Also, the white paper mentions that lots of partner onboarding and sharing will occur in 5G,
however, I question this because I'm not sure if companies will play nice with each other and be willing to share. 

\item {\it Interesting citations:}
Upon the beginning of the reading for the second half of the paper we touch on the idea of E-UTRA ~\cite{3gppeutra}.
E-UTRA is the NSA version of 5G that relies on the 4G core. I'm curious to find out more about the 
implications of using the 4G core for 5G devices.  

\item {\it Possible improvements:} 
No many improvements for the latter half of the paper. However, I would have liked to have seen a more in depth
take on 5G orchestration. Furthermore, I think they could have formatted their white paper a bit better in section
4 by providing subsections. By effectively organizing their subsections they have a better chance of delivering 
their main points to the readers. 

\item {\it Future work:} 
I had a few ideas of future work relating to the latter half of this white paper. For example, how effective will the NSA
5G phase conversion be with regards to latency and data speeds. A measurement study could be done for this. 
Secondly, I was thinking of testing and checking handover security and speeds when switching from a 4G to
5G network or vice versa. 

\end{itemize}

{
  \small 
  \bibliographystyle{acm}
  \bibliography{biblio}
}
\end{document}

























