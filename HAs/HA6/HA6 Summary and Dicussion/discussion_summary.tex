\documentclass[letterpaper,twocolumn,10pt]{article}
\usepackage{epsfig,xspace,url}
\usepackage{authblk}


\title{CS 6480: Class discussion summary\\
HA 6.b\\}
\author{José Monterroso}
\affil{School of Computing, University of Utah}

\begin{document}

\maketitle
\section*{Discussion summary}

\begin{itemize}

\item {\it Summary:} 
We touched on the fact that this is a magazine article, and therefore means that it will be shorter in length
and targeted to a specific field/topic. I also wasn't aware that SIGCOMM is one of the most competitive 
places to submit your papers to. We then discussed the goals and the details behind P4. Pretty much P4
is trying to improve on OpenFlow by allowing the creation of match+actions. And the way they propose to 
do this is through the use of programmable switches. However, P4 will most likely not be adopted because 
of the dramatic radical change that requires programmable switches. We spent a lot of time discussing the 
figures and the process that P4 takes to work. Going over the figures made me realize that I didn't really
understand a lot of the things being done in P4. 

\item {\it Strengths and weaknesses:} 
It was bit difficult to gauge the consensus in the class, but a few things were brought up. The 
example that dealt with top of the rack servers that was used in section 4 was really unclear. There was a bit of confusion 
in the class and in the paper not much was explained in-detail. Another weakness was the really weird small
related works section, why not just make into its own larger section. However, we did go over a lot of the
figures, and I do think they enhanced/improved the paper's content. Overall it was not a great paper but it had cool 
topics. 

\item {\it Connection with other work:} 
P4 is unique in its own right with regards to improving OpenFlow. But the underlying idea of being
able to program hardware is not. This paper relates to one of our previous papers OpenNetVM~\cite{opennetvm}
without getting into specifics this paper deals with being able to program the underlying hardware as well.

\item {\it Future work:} 
This time around we had a little class discussion for potential future work that could be done so I 
have lots to discuss. This paper revolves around the idea of programmable switches. So it was brought up
that we might be able to play around with open source programmable switches and might even be able
to do an analysis paper on the quantitive analysis of programmable switches. Improving on P4 we could
work on a GUI or IDE that makes the use of P4 more available and easy to learn/teach. On the more 
abstract side we could create a platform/hardware independent complier although this might not be 
feasible given the limits of hardware. 

\end{itemize}


{
  \small 
  \bibliographystyle{acm}
  \bibliography{biblio}
}

\end{document}

\begin{comment}

*Magazine article 
	=> CCR (communications review) 
	=> Sigcomm is the most competitive 
	
*Improvement of OpenFlow

Bearfoot Networks sell switches that can do this kinda stuff 

Talked about original openflow 

In P4 you have a program that specifies how those tables and match+action tables behave 

*Figure 1 is a diffrent way of getting a switch to do something but they are still using the Openflow part 
	=> using openflow on a P4 switch rather than OG switch 
	

*P4: Packet proccessors => Directly related to the paper from monday (OpenNetVM)
	=> Relying on proogrammable haardware to do this 
	
	
Reconfigurability in the field
	=> Programmers should be able to add or change the match+action of openflow
Protocol independence
	=>  Not tied to any type of networks
	
Target independence 
	=> Similar to like any programmin langugue you should worry about it not running it should just work
	
The swtiches through the P4 program are programable

Figuere2
Parser => Parsing the packet headers
Match+action

The target independence only gets you so far if your hardware can support it 

Figure 3 graphs get generated by the compiler 

*This paper is not very clear; especially with the example
	=> Example confusion with top or rack switch and core switches 

*Dramatic radical change that requires programmable switches 

How much can you push into hardware
	=> can i push moore specific processing in hardware??
	
*Not a great paper but cool topics 

*Future Work Ideas
	=> programmable switches
	=> Quantitative analysis of Programmable switches
	=> GUI, P4 IDE 
	=> Working Demo of P4; Already done
	=>  platform independent compiler (might not be feasible)
	=>  P4 for other hardware devices that might not be networking 





\end{comment}

























































