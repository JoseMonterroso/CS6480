\documentclass[letterpaper,twocolumn,10pt]{article}
\usepackage{epsfig,xspace,url}
\usepackage{authblk}


\title{CS 6480: Paper reading summary\\
HA 21.a\\}
\author{Jose Monterroso}
\affil{School of Computing, University of Utah}
\begin{document}
\maketitle
\section{Hiding in plain signal: physical signal overshadowing attack on LTE}

Paper discussed in this summary is ``Hiding in plain signal: physical signal overshadowing attack on LTE'~\cite{plainsight}.

\subsection{First pass information}
\label{sec:first}
\begin{enumerate}

\item {\it Category:} 
This paper is an analysis of an existing system as well as a description of a research prototype. The existing system 
that is being analyzed is 4G LTE security, while the research prototype being described is the signal overshadowing attack.

\item {\it Context:}
The technical area of this paper relates to 4G LTE mobile networking security. Our paper on 5G ~\cite{5gwhite}
is the closes we have gotten to a paper discussing the 4G LTE network design. We have also seen papers in the
past that discuss a different attack vector on 4G LTE named aLTEr ~\cite{breakinglte}.

\item {\it Assumptions:}  
The authors assume that their paper is the first to present a signal injection attack. I believe this is accurate as they
should have done research to verify no others have published such works, furthermore, I haven't heard or seen a 
paper that does such an attack.

\item {\it Contributions:} 
The papers main contributions are as followed. First, they produce the first signal overshadowing attack on LTE.
Secondly, they demonstrate the practicality and stealthiness of the SigOver attack via a real world experiment.
Thirdly, they present novel attack scenarios and analyze their implications in experiments. Lastly, they investigate
prevention and detection strategies against the SigOver attack. 

\item {\it Clarity:} Although this paper's length is longer than usually it does appear to be well written.

\end{enumerate}
\subsection{Second pass information}
\label{sec:second}
\begin{itemize}

\item {\it Summary:} 
The authors of this paper present the world's first signal injection attack that exploits the fundamental weaknesses 
of broadcast message in LTE and modifies a transmitted signal over the air. The first section after the introduction 
provides us with the needed background information of the LTE network architecture and the essential procedures 
of radio connection establishment, mobility management, and security setup between a device and an LTE network.
The next section describes the attack model, the description of the SigOver attack and a comparison with with a 
fake base station. The authors mention that an adversary can inject malicious messages into the victim UE(s) by
overwriting the legitimate messages. This can be done by carefully crafting a message that overlaps a legitimate
message with respect to time and frequency. In principle the SigOver attack leverages the capture effect, where 
the stronger signal is decoded when multiple simultaneous wireless signals collide in the air. Next, the Authors 
perform a SigOver attack in the wild and analyze the reliability of the attack. Their setting is within a university 
basement and office. They used an LG G7 ThinQ smartphone with SnapDragon845 which was the latest Qualcomm
LTE chipset at the time of this paper's release. We find that the SigOver attack demonstrated a 98 percent success 
rate when compared with the 80 percent success rate of the attack achieved by the fake base station. Next, the authors
present various attack scenarios and implications for each. Such attacks exploit paging and SIB. In section 6 the authors
discuss two possible defense strategies against the SigOver attack. Such solutions revolve around digitally signing all 
broadcast signals by using the public key infrastructure. Furthermore, you can detect the SigOver attack because
it leverages the changing nature of the physical signal during the processing of the overshadowing signal. Lastly,
the authors discuss related works before arriving to a conclusion.

\end{itemize}
\subsection{Third pass information}
\label{sec:third}
\begin{itemize}

\item {\it Strengths:} 
I really enjoyed the introduction because it presented me with all the high level overview information that you would
need from the paper. I also thought the background section gave a good overview of how LTE manages the UEs, eNBs,
and EPCs of the networks. I like how they had potential competitions and explained why the SigOver attack is better, it 
really brings up the attacks value. I really liked the use cases section which they called the attack scenarios and implications
section because it brought up the valid idea that their attack method is valid and can be used differently. I though that 
that way they described the potential solution while also included the deployment and technical challenges made it
seem like these guys really though everything out pretty well.

\item {\it Weaknesses:} 
Some of the little sub-sections within the background sections were a bit confusing and I wish they would have been more 
specific. 
They mention they implement their attack based on the pdsch enodeb and add a custom-built received function for time
synchronization I wonder if this can be generalized to other eNbs but they don't mention any others. All in all, I thought
that this was a good paper that didn't have many weaknesses. 

\item {\it Questions:}
I'm still a bit confused about the frequency synchronization aspect especially when they mention the oscillator and how 
they compensate for it.

\item {\it Interesting citations:} 
I always seem to find security papers really interesting as the majority of the things that people write about are loopholes
or small very detailed points of access. I like reading about the creativity and ingenuity people have to go through to 
find these security flaws. I find that this paper was full of security references, however the spoofing attack ~\cite{spoofing}
caught my interest as they appear to investigate the requirements for a perfect spoofing attack.

\item {\it Possible improvements:} 
This paper was a bit long for my taste but I realize that the length was generated due to the background and the 
specific and detailed nature of the author's writing. 

\item {\it Future work:} 
The authors mention checking if SigOver is possible for 5G. I would also be curious if this method can apply to 
all base stations in the real world. I'd be interested in seeing if overshadowing a signal can be applied to any 
other forms of communication theory. 

\end{itemize}

{
  \small 
  \bibliographystyle{acm}
  \bibliography{biblio}
}
\end{document}






