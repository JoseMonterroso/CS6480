\documentclass[letterpaper,twocolumn,10pt]{article}
\usepackage{epsfig,xspace,url}
\usepackage{authblk}


\title{CS 6480: Class discussion summary\\
HA 1.b\\}
\author{Your Name}
\affil{School of Computing, University of Utah}

\begin{document}

\maketitle
\section*{Discussion summary}

\begin{itemize}

\item {\it Summary:} Summarize the class discussion of the papers. Note in
particular anything that came out of the discussion that you missed in your
reading, or that made you change your mind. 

\item {\it Strengths and weaknesses:} Note the consensus
in the class, or your own new insights, regarding the strengths and
weaknesses of the paper(s).

\item {\it Connection with other work:} Describe any connections that were made in
the class discussion to other papers (with citations). (Or connections that you realized
as a result of the class discussion.)

\item {\it Future work:} Briefly describe any possibilities
for future work that came out of the discussion and/or were triggered
in your mind because of the discussion.

\end{itemize}
{
  \small 
  \bibliographystyle{acm}
  \bibliography{biblio}
}

\end{document}

\begin{comment}
Sigcomm - very important paper

In mobile network not knowing things is worse than in other networks 

Do you consider the core network the edge?
	=> 5G is moving the core to the edge; maybe that is why they are alluding to
	=> Kinda caught me off guard 
	
Measurement methodology for a measurement paper 
	=> their methodology is not that interesting 
		=> doesn't bring any new cool insights into measurements 
	=> most XCAL-Mobile one of the most interesting tool
		=> this tool was pretty crucial to their study 
		
TDD => One frequency can use TDD for time slots ; Send and receive , share the same spectrum  
FDD => uses two different radio frequencies for transmitting and receiving 

RSRP - receive signal receive power
	=> the power of the signal received 
	
if RSRP is less than -140 dBm the phone will not even attempt to communicate 

Look at slides for dBm useful cases and explanation 


Fig 1 B - Bit rate associated with one 5G cell
	=> line of sight, how far this would work without interference 
	
CDF - use this to show your results 
	=> 

Handover
	=> Based on measurements and algoorithm the UE will request a handover
	=> in 5G in NSA it goes down to 4G and then handover 
	
The UE does measurments to a from other base statioons
	=> then it sends its measuremnts to the its current base satation
	=> the decision is taked by the base station 

Want to know the new signal strength is better for a while

RSQR Gap - You have made this decison to handover are you better of with the new base station?


 

\end{comment}





























































