\documentclass[letterpaper,twocolumn,10pt]{article}
\usepackage{epsfig,xspace,url}
\usepackage{authblk}


\title{CS 6480: Paper reading summary\\
HA 11.a\\}
\author{José Monterroso}
\affil{School of Computing, University of Utah}

\begin{document}

\maketitle
\section{Understanding Operational 5G: A First Measurement Study on Its Coverage, Performance, and Energy Consumption}

Paper discussed in this summary is ``Understanding Operational 5G: A First Measurement Study on Its Coverage, Performance, and Energy Consumption''~\cite{measurementstudy}.

\subsection{First pass information}
\label{sec:first}

\begin{enumerate}

\item {\it Category:} 
This paper is a measurement paper on operational 5G. The authors of this paper measure 5G coverage, 
performance, and energy consumption. 

\item {\it Context:} 
The technical area of this paper deals with 5G networking. Specifically it deals with the coverage, performance,
and energy consumption of 5G devices. This is the first measurement paper we have read in this class. However,
It does greatly relate to our previously read paper on the 5G infrastructure ~\cite{5gwhite}.

\item {\it Assumptions:}  
The Authors of this paper built a reliable toolset for the various measurements they do with coverage, performance,
and energy consumption. They assume that their technology is reliable and produces accurate data. I'm honestly 
not quite sure how much I can trust their stuff, however, this paper was published in ACM which makes me believe
that they are reliable and that their assumptions are valid. 

\item {\it Contributions:} 
The paper's stated claims are as follows. First they quantify characteristics of 5G's coverage in comparison to 4G.
Second they identify an alarming TCP anomaly that severely underutilizes 5G capacity. Third they do a breakdown
analysis of the 5G end-to-end latency which pinpoints the bottleneck. Fourth they implement and profile a 5G 
immersive media application to explore the feasibility and underlying challenges. Fifth they provide a detailed 
account of the power budget on 5G smartphones. And finally sixth they have released their dataset and measurement
tools to the public. 

\item {\it Clarity:} 
From what I have read this paper does appear to be written well. 

\end{enumerate}

\subsection{Second pass information}
\label{sec:second}

\begin{itemize}

\item {\it Summary:} 
This paper is discussing the measurements found from studying 5G coverage, performance, and energy consumption.
After the abstract and the paper's introduction, they then next discuss the measurement methodology behind their
paper. They describe how most of their experiments focused on a campus where 6 5G base stations are deployed
and the campus is surrounded by tall buildings, trees and heavy human activity. Furthermore, the campus has a 
NSA deployment of 5G. The authors use ZTE Axon10 pro, HUAWEI Mate20 X, and a HUAWEI Mate30 pro as UEs.
Within this section they then go in-depth of the measurement tools they used and created for this paper. In section 3 
the authors give us the lay of the land, by discussing the RSRP of the 4G and 5G networks around campus. As well 
as provide us with cell coverage information around their campus indoors and outdoors. We then touch on handover
and its latency. Next in section 4 we discuss transport layer throughput, TCP throughput during handoff and end-to-end
latency. In section 5 we touch on 5G application performance, examining the application QoE under 5G. Lastly, the 
authors run a microscopic analysis of the power management under the 5G NSA.

\end{itemize}

\subsection{Third pass information}
\label{sec:third}
\begin{itemize}

\item {\it Strengths:} 
From the beginning we get a clear understanding of what this paper is about, and what direction this paper will take,
we even get the results in the beginning. That is why I think that the abstract and introduction are a big plus in this paper.
Furthermore, this paper used a lot of technology (UEs) to do their measurements. In all cases we got a clear description of
the technology used as far as names and some specifications. Lastly, I really thought the format of the sections and subsections
gave the paper a professional look and feel.

\item {\it Weaknesses:} 
The authors only used what appear to be Huawei technology (phones, cloud server). I believe if they expand their 
devices they could have acquired non-bias Huawei data. I really enjoyed their tables and figures but the placement 
was really poor. All the tables and charts look squished and pressed for space. Lastly, a bit of a nitpick but they only
got data from 5G NSA, I know it's hard to find standalone 5G but that would greatly add to their paper. 

\item {\it Questions:} 
They used a lot of acronyms (PCI, RSRP, RSPO, SINR, COL, MCS) that I am not at all familiar with, so I will be doing
some research. I also have a few small questions on how handover works. Lastly, I didn't realize how expensive 5G gNBs
are. 

\item {\it Interesting citations:} 
It is because of my final project that is on radio propagation, that I found this paper on 5G performance
~\cite{5gnetworkper} to be an interesting read. 

\item {\it Possible improvements:} 
I believe that if they addressed the issue with the graphs being crunched together and changed their layout a bit
their paper would improve. Also they could have used other technology along with Huawei to improve their results. 

\item {\it Future work:} 
I think what they did could be expanded to the standalone 5G (5G core). This paper also covers a lot of little things, 
maybe we could choose one and dive a little deeper on it. We could even use their new technology that they developed
to measure other 5G bands. 

\end{itemize}

{
  \small 
  \bibliographystyle{acm}
  \bibliography{biblio}
}
\end{document}


































