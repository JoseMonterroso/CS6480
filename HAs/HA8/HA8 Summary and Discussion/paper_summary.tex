\documentclass[letterpaper,twocolumn,10pt]{article}
\usepackage{epsfig,xspace,url}
\usepackage{authblk}


\title{CS 6480: Paper reading summary\\
HA 8.a\\}
\author{José Monterroso}
\affil{School of Computing, University of Utah}

\begin{document}

\maketitle
\section{Tussle in Cyberspace: Defining Tomorrow's Internet}

Paper discussed in this summary is ``Tussle in Cyberspace: Defining Tomorrow's Internet''~\cite{tussle}.

\subsection{First pass information}
\label{sec:first}

\begin{enumerate}

\item {\it Category:}
This paper appears to be a mix of an analysis of an existing system, as well as a description of a research 
prototype. The analysis part deals with the internet itself and current issues at the time this paper was published.
But the authors also propose some design principles that are intended to be accommodated within the future 
internet. 

\item {\it Context:} 
The technical area of this paper relates to internet architectures. However, this paper seems to target the 
sociopolitical attributes of the internet. This paper references a few of the main concepts and ideas from our
previous paper on the DARPAnet~\cite{darpa}.

\item {\it Assumptions:}  
The authors assume that the internet today is owned by multiple stakeholders that have different interests
and are pursuing their own particular interests. I agree with their assumption because at the end of the day
ISPs are businesses and businesses need to make money. Another assumption they bring up was that the 
creators of the internet shared a common goal. Now this assumption is true with regards that everyone 
wanted to connect various networks together but I hardly agree with the fact that they shared a consistent 
vision and common sense of purpose. From our previous paper it seemed that the internet got put
together with whatever worked best at the time.

\item {\it Contributions:}
As the internet continues to grow, its development imposes new requirements. The authors of this paper 
contribute by designing new internet design principles, as well as addressing the tussle and providing guidance
that may be of value in accommodating the tussle. 

\item {\it Clarity:} 
From what I have read this paper seems to be well written, it even includes some fancy words in the abstract. 

\end{enumerate}

\subsection{Second pass information}
\label{sec:second}

\begin{itemize}

\item {\it Summary:} 
The paper introduces the idea of tussle and how it clouds the judgement of different internet stakeholders. 
In essence, tussle is forcing the internet to take the form of greedy stakeholders, however, the authors argue
that it's the job of engineers and researchers to overcome the tussle and produce great results without being
clouded by the tussle nor produce more tussle. The authors go into detail with regards to the types of tussle, 
some of which are economic, others deal with trust and openness. The authors then discuss end to end 
arguments. These arguments deal with innovation, reliability and transparency as well as the separation
of policy and mechanisms. Lastly, the authors of the paper list a few things that designers should follow. 
The main idea in this section was to consider the tussle that you will unearth with your creation and how
you can potential analyze and reduce the tussle. To conclude, tussle will always occur when we create 
things and in some cases it's nearly impossible to remove the tussle, however, we should do what we can
to work around it and do our best to alleviate the tussle. 

\end{itemize}

\subsection{Third pass information}
\label{sec:third}
\begin{itemize}

\item {\it Strengths:} 
I really do think the introduction is a strength of the paper. The introduction does a great job of introducing
the tussle, and why we should care to fix it. For every design principle listed, the authors provide a nice 
example that easily illustrates what they hope people to understand. I also like the format of the paper,
especially how the bulleted design principles are used to guide the papers discussion. Furthermore, I liked 
how the sections were built on each other.  

\item {\it Weaknesses:}
Although the paper did a great job of discussing tussle, I felt that it lacked concrete answers. Most of the 
design principles are open ended. The authors rely to many times on the idea that each scenario 
is different and thus a direct answer can't be provided. However, I would argue that if you are considering 
a change to the internet you should be specific and bold with your answers. I also felt that the last section
on designers really lacked content. I mean they provide an example but hardly list any guidelines or lessons
for future designers.    

\item {\it Questions:} 
The paper mentions IP QoS design, and the ToS bits that are used in QoS. I not sure what any of this means,
so I will do some research. 

\item {\it Interesting citations:} 
This is the second paper we have read regarding internet architectures and in both papers we find David Clark 
to be the author. He actually references a few of his own previous works. Looking at the title I think "Rethinking the 
design of the Internet: The end to end arguments vs. the brave new world"~\cite{newworld} looks like an interesting 
read.

\item {\it Possible improvements:} 
I honestly think that their first four sections were great. But I feel towards the end of the paper, during the fifth
and sixth sections, the authors lose their motivation. The fifth section is lacking new information if anything it feels
like a conclusion. I also feel that the conclusion needs to be redone. They talk about cyberpunk, privacy,
and bureaucrats in the conclusion when the paper slightly related to theses topics.

\item {\it Future work:} 
This paper has opened my eyes to look at all the technology that I have developed or will develop and how it 
can produce tussle. Future work for the authors could relate to developing application design guidelines that
will help designers avoid pitfalls and deal with the tussle of success.

\end{itemize}

{
  \small 
  \bibliographystyle{acm}
  \bibliography{biblio}
}
\end{document}




























