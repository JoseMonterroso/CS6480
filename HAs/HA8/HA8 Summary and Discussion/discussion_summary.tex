\documentclass[letterpaper,twocolumn,10pt]{article}
\usepackage{epsfig,xspace,url}
\usepackage{authblk}


\title{CS 6480: Class discussion summary\\
HA 8.b\\}
\author{José Monterroso}
\affil{School of Computing, University of Utah}

\begin{document}

\maketitle
\section*{Discussion summary}

\begin{itemize}

\item {\it Summary:}
Our discussion of this paper was pretty straight forward. This paper talks about future internet philosophies.
This paper was definitely different from papers we have read in the past because it's less technical and more
philosophical. In the paper, tussle is defined to be argument points among users and stakeholders of the internet.
These tussles do not have a final end state, they simple evolve. A key takeaway from our discussion was that in order
to reduce tussle you should design your technologies to give the users choices.

\item {\it Strengths and weaknesses:} 
I think this paper does a good job of bring up valuable points. These points can help the readers understand
how their developed technologies or even presence on the internet creates tussle. However, this paper is a bit
old and does contain various outdated examples. 

\item {\it Connection with other work:} 
The way the authors of the paper define tussle, you could associate tussle with all aspects of technology.
You'll always find some kind of argument among users. The DARPAnet~\cite{darpa} paper is a bit of the 
opposite of what the authors discuss in the paper. For example, the creators of the internet envisioned a future
world where all the previously existing networks were connected to form a vast array of communication. But
now that we've had years with the internet we notice that the way we use the internet has drastically 
changed to a more commercialized approach. This has made users reconsider the internet's architecture
to introduce security aspects. The tussle paper gives us some of the principles we should consider for a better
internet architecture.  

\item {\it Future work:} 
Personally I was considering that in all future work developed by us, we could consider the tussle that 
surrounds what we will be creating. The fundamental features that lower the gravity of the tussle would be 
to incorporate choice in our developed technologies. When it comes to security aspects regarding endpoints
we could build a fair system for authenticating endpoints, and potentially incorporate block chain technologies
to create transparency and accountability within the internet. Lastly, the authors mention how DNS provides 
significant tussle. If we could some how isolate trademarks from DNS or even verify that someone isn't buying
trade marked names, we could potential solve the evolving DNS tussle. 

\end{itemize}
{
  \small 
  \bibliographystyle{acm}
  \bibliography{biblio}
}

\end{document}

\begin{comment}

SIGCOMM conference paper 

*Abstract
	=> This paper talks about future internet philosophies 
	
*Introduction
	=> common vision for internet connection
	=> but not a common vision on who the internet should be work
	=> They wanted to interconnect the research networks but not with the goal of introducing new apps for the future 
	
*This felt like a philosophy paper 
	=> less technical
	=> but more philosophical 
	
*These tussles do not have a final end state, they just evolve 

*Section 2
	=> divergence in interest lead to tussle, so that we must accommodate for this future 
	=> different tussles happen they won't break the creation 
		=> modularization
		=> design for choice allow ppl to have choice 

	=> QoS using the ToS bits you have choice for QoS

Design for choice
	=> fear of compititon 
	=> allow for choice but it has drawbacks because it results in tension 
	
*Introducing SDN 
	=> Tussle of choice and configurtion: helps 
	=> doesnt really apply to the general internet 
	=> USERs don't mess with SDN policies or rules 
	
Implications
	=> the interface is open stays the same but the implementation is not open 
	=> ISP example: ISPs are competitors but they have to cooperate 
		=> reason why BGP interaction has not really involved 
	=> tussle between provider and customer
		=> price points vs competition
	There is no such thing as value-neatural design 
		=> no matter what we choose to create or use mooney from a compnay will have these things embedded in it 
			=> name carries with you 
			
*Visible exchange of value
	=> ppl use facebook platform and FB makes money 
		=> Google search 

Tussle Spaces 
	=> money 

Municipalities vs companies
	=> fiber networks 
	
Trust
	=> issues with identity and accountability 
	
Endpoints 
	=> services offered by the core should be general services 
	
*Future Work
	=> isolate trademarks from DNS 
	=> building a fair system for authentication of endpoints 
		=> secure problem
	=> make a decentralized platform 
	=>  policing algorithms 
	=> trust using block chain technologies 
\end{comment}


























































