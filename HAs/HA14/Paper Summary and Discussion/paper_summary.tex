\documentclass[letterpaper,twocolumn,10pt]{article}
\usepackage{epsfig,xspace,url}
\usepackage{authblk}


\title{CS 6480: Paper reading summary\\
HA 14.a\\}
\author{José Monterroso }
\affil{School of Computing, University of Utah}

\begin{document}

\maketitle
\section{XIA: Architecting a more trustworthy and evolvable internet}

Paper discussed in this summary is ``XIA: Architecting a more trustworthy and evolvable internet''~\cite{xia}.

\subsection{First pass information}
\label{sec:first}
\begin{enumerate}

\item {\it Category:}

This paper is a description of a research prototype. The research prototype in this paper is the eXpressive 
Interent Architecture (XIA). 

\item {\it Context:} 

The technical area of this paper relates to internet architectures, in fact they cover similar topics to that of the 
DARPAnet ~\cite{darpa} paper we read a few weeks back. This paper also relates to our paper on tussle ~\cite{tussle}
because the authors of XIA bring up similar ideas when they discuss the goals of XIA. On a smaller note this 
paper discusses the narrow waist, which we first learned in our NDN~\cite{ndn} paper.

\item {\it Assumptions:} 

They make the assumption that with our current internet, it is difficult to provide a clean path for incremental 
deployment. I think this assumption is true, because of the examples they give with regards to IPv6 and 
internet security which makes me believe this is a true problem in our current internet.

\item {\it Contributions:} 

They don't have any stated contributions but rather have goals with XIA for the network layer. Their goals are 
they want to be trustworthy, support long-term evolution of usage models. support long-term technology
evolution, and support explicit interfaces between network actors. 

\item {\it Clarity:}

From what I have read of this paper so far, it appears to be written well. 

\end{enumerate}

\subsection{Second pass information}
\label{sec:second}
\begin{itemize}
\item {\it Summary:} 

The eXpressive Internet Architecture (XIA) is an internet architecture with native support for multiple principals
and the ability to accommodate new principles over time. XIA keeps several current internet features but seeks
to modify and extend the network layer. The three main design principles of XIA are: communication between 
diverse entities, communication operations should have intrinsic security, and lastly the architecture must 
support flexible addressing. The XIA architecture has four basic XIA identifier types, each with their own 
intrinsic security properties. The four are: Host XIDs, Service XIDs, Content XIDs, and Network XIDs. For 
flexible addressing, XIA uses a Directed Acyclic Graph of XIDs as addresses. Although XIA has not been 
fully fleshed out in code, there have been promising attempts of using the ideas of XIA to make things
better. For example, the use of Click on Linux gave them promising results for pushing the limits of 
extremely large scale flat host, service, or content forwarding tables. XIA can be incrementally added to
our current IPv4 network through the use of IPv4 tunneling. The authors also designed FCP a novel
congestion control framework. The SCION architecture features scalability, control, and isolation for secure
and highly available host-to-host communication. Lastly, XIA offers economic incentives for caching and 
privacy through the use of XIDs. 

\end{itemize}

\subsection{Third pass information}
\label{sec:third}
\begin{itemize}

\item {\it Strengths:} 

What made me understand XIA the most was when they listed their goals in the introduction. This helps the 
reader get a good understanding of what the authors are hoping to do. They have prototypes on GitHub which is 
always a great thing. And lastly their figures helped me visualize their ideas, especially the parts that related to the 
DAG. 

\item {\it Weaknesses:}

I felt that there abstract was a bit weak, after I read it I had no idea what the paper was going to be about. Furthermore,
a lot of XIA's ideas are high level. I understood the idea but failed to see how it would work or be better than what we
currently have. I'm also confused with all the various aspects that tie into XIA. It appears that they made all these 
different frameworks and technologies, and now they want to combine them all and call them XIA together. Lastly, I 
would have liked to have seen some data to back up their claims, especially their FCP claims. 

\item {\it Questions:}

Throughout the paper they list a bunch of technologies that they have developed that can be used in XIA.
But they mention that XIA hasn't been fully fleshed out, so I question if they will be able to fully integrate 
all these aspects together in the end. 

\item {\it Interesting citations:} 

They make a huge case for Serval~\cite{serval}, and how they were able to port a linux implementation 
of Serval to XIA and leveraged XIA's features to improve Serval. So I would like to give Serval a read. 

\item {\it Possible improvements:} 

What would have made this paper better would have been some use cases of XIA as a whole 
rather than the various parts of XIA that can be used. I was missing the big picture of XIA. 

\item {\it Future work:} 

Potential future work could be XIA for mobile networks. Their paper also lacked data to back up most of their claims
so I think a measurement paper would be in order to gauge XIA and its capabilities. 

\end{itemize}

{
  \small 
  \bibliographystyle{acm}
  \bibliography{biblio}
}
\end{document}





















