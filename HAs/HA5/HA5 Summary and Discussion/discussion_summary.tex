\documentclass[letterpaper,twocolumn,10pt]{article}
\usepackage{epsfig,xspace,url}
\usepackage{authblk}


\title{CS 6480: Class discussion summary\\
HA 5.b\\}
\author{José Monterroso}
\affil{School of Computing, University of Utah}

\begin{document}

\maketitle
\section*{Discussion summary}

\begin{itemize}

\item {\it Summary:}
The first big topic that was discussed was that this paper was entered to a workshop.
I missed that workshop papers would generally be shorter, and will be a bit different than
papers we have read in the past. Next we jumped into the abstract, specifically the part that mentions their 
goal. A point was brought up stating that it's a bit odd for them not to elaborate on their papers goal in the later
sections. Even though they do in my opinion meet their goal, it's a bit odd that they didn't bring their goal up as a point
of reference or motivation in the later sections. In the Introduction section one thing that I missed was the 
difference in the use of network functions compared to other papers we have read. In previous papers
there were more complex processes for NFV compared to this paper that dealt with packet forwarding. 
The key characteristics of OpenNetVM was the use of a polling
system, bypassing the Kernel, and Zero copy. What I missed here was that in order for you to bypass 
the Kernel and do things the way you want them to be done you need to do a lot of low level things yourself. For
example memory management of packets. Lastly, I'd like to mention the class confusion on the similarities
and differences between service chains and flows. The paper made them seem interchangeable, but a few
points were brought up that they weren't.

\item {\it Strengths and weaknesses:}  
The consensus of the class seemed to favor the paper. A strength 
that was brought up was the abstract. It was a strength in the sense that it described the problem, offered
a solution and talked a little about the outcome. We also thought their related works was a strength because
it did a good job of presenting competing apps and how OpenNetVM solved their flaws.  A few weaknesses that 
were discussed related to how the design principles were very similar to the design requirements for NFV 
in previous papers we have read. This lead us to believe that they were a bit lazy in their approach. 
Another weakness was brought up that their execution was a bit too narrow. They took all these different
pieces and put them together. A thought was brought up on how many new things they really did or
contributed to NFV research.

\item {\it Connection with other work:} 
There were a few general connections to other papers. Like NFV from our previous paper, and using Docker
to create containers. DPDK was also brought up from one of our previous papers on NFV.

\item {\it Future work:} 
It was brought up that their orchestration and containers were all located within a single node. Ideally this could be
expanded to orchestrating multiple nodes.

\end{itemize}


\end{document}

\begin{comment}
*This paper is a workshop paper 
	=> A little different from research paper
	=> Shorter paper 
	=> Sitcom workshops are 
	=> organizations attempt to explore type 
	
	
*They followed through with their goal
	=> Emphasized in the abstract, but rest of paper its hardly mentioned
	=> they followed through with providing open source code 
	
*Pretty nice abstract 

*Introduction emphasizes diffrent NFV processes 
	=> more complex process for NFV in previous papers but not in this one; just packet forwarding 
	
*Key characteristics (Get Speed, but everything is pretty RAW; do a lot yourself) 
	=> polling system
	=> bypass the kernel 
	=> Zero Copy: copy from NIC only once 
	
*Was the motivation with the execution to narrow
	=> split feelings in the class 
	=> Taken all the pieces and put them together (How much new have they done?) 

*Design principles were very similar to last week papers
	=> Design principle = the way you think of what you wanna do 

*Follow up work from NetVM to openNetVM but switched from VM to Containers 
	=> a necessary follow up to previous work 


*Yes they have a manager but the manager is associated with a single node
	=> orchestrating containers within a single node 
	
Related works was good 
	=> Really dense area but want to be pretty explict with what they are using and adding 
	
Went over Fig 1 => in depth 
	=> Memory management (Bypasses the Kernel)
		=> allocates the appropriate packets coming in into the appropriate subset for containers 
	=> NFLIb
		=> It registers with a mangare, packet show up, then using callback function telling it what it needs to do
	
		
	
*Containers are just processes or groups of processes running 
	
*Service chains and flows seem similar in the way they describe them (STUCK ON THIS PART)
	=> flows appear to be finer grain than a service chain 


\end{comment}






















































