\documentclass[letterpaper,twocolumn,10pt]{article}
\usepackage{epsfig,xspace,url}
\usepackage{authblk}


\title{CS 6480: Paper reading summary\\
HA 5.a\\}
\author{José Monterroso}
\affil{School of Computing, University of Utah}

\begin{document}

\maketitle
\section{OpenNetVM: A Platform for High Performance Network Service Chains}

Paper discussed in this summary is ``OpenNetVM: A Platform for High Performance Network Service Chains''~\cite{opennetvm}.

\subsection{First pass information}
\label{sec:first}

\begin{enumerate}

\item {\it Category:} 

This paper is a description of the OpenNetVM research prototype. OpenNetVM is a platform for high performance 
network service chains. 

\item {\it Context:}

The technical area of this paper relates to network function virtualization (NFV), and how we can get high 
performance for service chains. This paper relates to a few papers we have read in the past for example
last time we read about NFV~\cite{nfv}, they also reference Docker and its containers~\cite{containerization}.

\item {\it Assumptions:}  

A major assumption that the authors of the paper make is that they assume that NFV research can grow
with the development of a flexible and efficient platform enabling high performance NFV implementations.
I think their assumption is valid because when you implement something that everyone can use
without having to make the user redefine nor create difficult things, that area of research expands because it becomes
simpler for all those working in it. 

\item {\it Contributions:} 

The paper contributes by combining multiple NF techniques into an efficient and easy to use platform that is available
 for the community to use.  

\item {\it Clarity:} 

From what I have read so far I do believe that this paper is well written. 

\end{enumerate}

\subsection{Second pass information}
\label{sec:second}

\begin{itemize}

\item {\it Summary:} 

The writers of this paper created OpenNetVM to produce a flexible and efficient platform enabling high 
performance NFV. OpenNetVM based on the NetVM architecture runs network functions in lightweight 
Docker containers, and provides load balancing, flexible flow management and service name abstractions.
Furthermore, OpenNetVM uses DPDK for high performance I/O, and efficiently routes packets through 
dynamically created service chains. OpenNetVM is split between the NF Manager which interfaces with the 
NIC, and the NFLib which provides the API within an NF to interact with the manager. These two components
help OpenNetVM to become flexible and fast. Lastly, OpenNetVM is evaluated to achieve throughputs of 68
Gbps when load balancing across two NF replicas, and 40 Gbps when traversing a chain of five NFs.

\end{itemize}

\subsection{Third pass information}
\label{sec:third}
\begin{itemize}

\item {\it Strengths:} 

I really liked the abstract. The abstract caught my attention, presented the problem they are trying to solve, and 
offered their solution. It was just enough information that made me want to read more, but not to much information 
that told me the whole story. Furthermore, the Background \& Related Works section does a great job of providing 
similar technologies, while also downplaying them because of their flaws. This results in the authors telling us why
OpenNetVM is better because it solves these problems. I also liked their evaluation section, I think they did 
a decent job of leveling the playing field between the two systems that were being compared. 

\item {\it Weaknesses:} 

I really didn't like the last half of the introduction. It was a bit odd to just mention things about NF. Maybe it would 
have been better applied to the Background \& Related Works section. I felt like the Evaluation section could have used more
content, I just didn't feel too convinced about OpenNetVM.

\item {\it Questions:} 

Honestly I do not have any questions. I think they did a great job of explaining OpenNetVM and its uses. 

\item {\it Interesting citations:}

Last time we spoke of the differences and striking similarities between SDN and NFV. This paper seems to
offer an interesting reference to network function virtualization in SDN and OpenFlow ~\cite{nfvsdn}

\item {\it Possible improvements:} 

I think if they fixed the weird second half of the introduction, and moved that content to a Background
section, the paper would improve. Furthermore, I wish they would have added more to the Evaluation section.
The Evaluation section was a bit short and lacked content in my opinion. 

\item {\it Future work:} 

They mention how small improvements like using RSS hash to simplify packet lookups can significantly
improve the performance. I guess future work related to OpenNetVM could be finding these little
improvements to help OpenNetVM performance. I would also be interested in doing some research
into service chaining. Service chaining seems to be very popular among NFV so I believe there could
be a better or even faster way of doing service chaining. 

\end{itemize}

{
  \small 
  \bibliographystyle{acm}
  \bibliography{biblio}
}
\end{document}























