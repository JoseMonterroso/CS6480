\documentclass[letterpaper,twocolumn,10pt]{article}
\usepackage{epsfig,xspace,url}
\usepackage{authblk}


\title{CS 6480: Paper reading summary\\
HA 4.a\\}
\author{José Monterroso}
\affil{School of Computing, University of Utah}

\begin{document}

\maketitle
\section{Network Function Virtualization Challenges and Opportunities for Innovations }

Paper discussed in this summary is ``Network Function Virtualization Challenges and Opportunities for Innovations''~\cite{nfv}.

\subsection{First pass information}
\label{sec:first}

\begin{enumerate}

\item {\it Category:} 

This paper is an analysis of an existing system. The system that the authors of this paper are analysis is Network 
Function Virtualization (NFV). 

\item {\it Context:} 

The technical area of this paper relates to virtualization. This paper doesn't specifically relate to other papers we
have read, however, virtualization is used in cloud computing~\cite{containerization,aview}. 

\item {\it Assumptions:}  

They assume that NFV is prospectively the unifying revolution among NFVs, SDN, and Cloud Computing, 
offering more revenue opportunities in the services value chain. I don't entirely agree with their assumption
that NFVs have more revenue opportunities because I didn't even know about NFVs existed before this paper.
I'm more familiar and believe that cloud computing is more popular and produces more revenue. 

\item {\it Contributions:} 

Their contributions are as follows. They introduce NFVs architectural framework, describe several use cases of NFV, 
they discuss the open NFV research issues and point out future directions for NFV.

\item {\it Clarity:} 

From what I have read I do think this paper is well written. The paper has good headings, and an explicit abstract and 
introduction, compared to some of the previous papers we have read in the class that do not.  

\end{enumerate}

\subsection{Second pass information}
\label{sec:second}

\begin{itemize}

\item {\it Summary:} 

The author's goal in this article is to bridge the gap by identifying critical research challenges involved in the
evolution toward NFV. NFV implements network functions through software virtualization techniques and runs
them on commodity hardware. The article discusses the creation of the Industry Specification Group for NFV
to achieve the common architecture required to support VNFs. We then learn of some of the technical requirements
of NFVs. Performance, manageability, reliability and stability, and security are some of these requirements. We then
learn about the architectural framework, and see some use cases related to NFV. Lastly, the authors discuss a few
research challenges and provide future directions of NFV.

\end{itemize}

\subsection{Third pass information}
\label{sec:third}

\begin{itemize}

\item {\it Strengths:} 

I liked the papers abstract, I think it did a good job of introducing me to NFV and the papers content.  
A strength in this paper is when they discuss the design and architectural framework. I felt that this section
really helped me understand NFV. Furthermore, tying in the Use Cases section further enhanced my 
understanding of NFVs through the use of real world examples. 

\item {\it Weaknesses:} 

Because this paper doesn't introduce a new idea, I can see why their related works section contains
NFV activity and implementations. However, I don't really agree with the idea that this is considered
related works. I think that they could have rather have a background section or have added parts to 
the introduction. Another weakness I found was in the Technical Requirements section. They tell us
to understand the maximum achievable performance of the underlying programmable hardware 
platforms. And from this information we can make a reasonable design for NFV. I feel that this is a lazy 
way of saying make sure you check the specs of your hardware before you buy it, rather than actually
providing a good technical requirement they claim to help with. 

\item {\it Questions:} 

The authors made it seem like a VNF forwarding graph was a big deal, but hardly went into detail. So 
I have some questions about VNF forwarding graphs. 

\item {\it Interesting citations:} 

I was interested in the idea of IPTV~\cite{iptv} so I decided to look into one of the references that the authors list. 

\item {\it Possible improvements:} 

I thought the paper did a good job of providing an overall introduction to NFV. However, I do wish
they would have gone into a little more detail in the design and architectural framework section. A 
lot of concepts and idea were brought up but not much was said about how these things are used 
or work together with other NFV technologies. 

\item {\it Future work:} 

As the paper mentions toward the end, efficient management and orchestration of VNFs is a 
challenging issue that not many people have found great solutions too. This paper is a bit old 
so I bet some people may have discovered new methods, but I do think this would be an interesting
thing to research. I've also never new IPTV was a thing, so I think this could be a potential future 
research subject. 

\end{itemize}

{
  \small 
  \bibliographystyle{acm}
  \bibliography{biblio}
}
\end{document}





















