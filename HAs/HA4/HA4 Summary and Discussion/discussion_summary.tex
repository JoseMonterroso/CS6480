\documentclass[letterpaper,twocolumn,10pt]{article}
\usepackage{epsfig,xspace,url}
\usepackage{authblk}


\title{CS 6480: Class discussion summary\\
HA 4.b\\}
\author{José Monterroso}
\affil{School of Computing, University of Utah}

\begin{document}

\maketitle
\section*{Discussion summary}
\begin{itemize}

\item {\it Summary:} 
We first discussed how this paper was an IEEE communication magazine article. What I failed to 
understand was that this paper would cover things at a higher level, and not include many citations. 
We then discussed how NFV implements network functions through software virtualization techniques
and runs them on commodity hardware. This is great because if you are a service provider you are 
not locked in, and have flexibility to get the best pieces together. We then slightly touched
on 3GPP, and IETF, and how they create and propose standards for engineering. We then went into to some details about 
performance and how software is competing with non-native hardware, and how with virtualization
you have other tenets and big overheads. When it comes to manageability we need to worry about 
service chaining. We also brought up the point that general purpose hardware does not have the same
reliability as a dedicated piece of hardware, therefore, we need to make the software more reliable. Lastly,
we discussed how orchestration, and shared resources for VNFs brings up security concerns. 


\item {\it Strengths and weaknesses:} 
The consensus for this paper was a bit split. We discussed how the paper did a great job of explaining NFV. We 
also noted that the abstract and introduction were really good sections that layout the contents of
the paper fairly well.  Lastly, we considered the Use Cases section to be 
good because it provided real world examples of NFV. On the topic of sections, the Related Works
section was poorly made and resembled more of a Background section. Another weakness that 
was brought up was that there really wasn't any good analysis of NFV. Finally, we discussed that 
the authors of the paper drop a lot of big concepts/words and didn't go much in depth. We felt this 
was a weakness because we didn't know what the word meant and why it's important with regards 
to NFV.

\item {\it Connection with other work:} 
This paper didn't really connect with any other papers we have discussed. One slight connection
that I can see is the use of virtualization in NFV, which is done in cloud computing and for microservices. 
Another small connection is how NFV needs orchestration tools just like how Docker needs orchestration
for larger complex multi-container systems. 

\item {\it Future work:} 
Future work that was brought up dealt with looking into the Linux New API (NAPI) and Intel's Data Plane 
Development Kit (DPDK) for clustered VNF instances. I also wasn't really aware of how important service
chaining is, so I believe this would be a good area to look into.

\end{itemize}
\end{document}


\begin{comment}
*IEEE communication magazine ** DIDN'T know this would mean the type of paper that this paper was (covered many things no detail)
	=> higher level, shorter, not that many citations 
	=> thank the editors guess editors 
	
*They bring up a lot of things but don't go into detail about it 

*NFV decouples software implementation of network function from the underlying hardware  	
	=> You can have multiple functions running on a piece of hardware
	
*NFV implements network functions through software virtualization technique and runs them on commodity hardware

*If you are a service provider you are locked in when you buy from jupiter and ciscoo
	=> but with NFV you are not limited to be locked in with software
	=> you can get the best pieces together
	
*The NFV provides flexibility, with software (Time to marked IDEA)
	=> target and tailored services 
	
*They bring up NFV issues
	=> Network performance
	=> How to smoothly migrate from the existing network infrastructure to NFV based solution 
	
	
*Related Works, not good

*3GPP => standard bodies for cellular mobile technologies
*IETF => Create and propose standards for engineering  (RFC => request for comments)
	=> less formal process but in practice they are standards 

*Technical Requirements
	=> Performance: corresponding physical version hardware is faster/better 
		=> with NFV we need virtualization 
			=> software competing with hardware, 
			=> virtualize: other tenets and OVERHEADS
		=> Clusters need to share everything so we need to distribute out load (distribute complexity)
	=> Manageability: 
		=> service chaining: somethings need to be in a specific place 
			=> have to go through things in the right order 
	=> Reliability and stability 
		=> general purpose hardware does have the same reliability as a dedicated piece of hardware 
		=> 5 nines of reliability
	=> Security: 
		=> orchestration and hypervisiours 
		=> all the new software we have to trust
	
**Read Linux New API and intel's data plane development kit 

Compose several VNFs together to reduce management complexity vs decompose a VNF into smaller functional blocks for
reusability and faster response time 


The two major enablers of NFV are indsutry-standared servers and technologies developed for cloud computing

SDN and NFV are mutually beneficial to each-other, enhance its performance, facilitate its operation and simply the
compatibility with legacy deployments 

A shift towards a service based network 

The complex stuff happens at the clouds
	=> makes the claim that the things in the home can be much simpler 
	=> more radical 
	=> do not go into detail what is the layer 2 device they talk about
	
	
NFV has more flexibility for migrations 

State needs to be migrated not just the software 

End to end principle: of the initial internet architecture that does not modify packets on the fly is no longer valid in current network
	=> complexity at the endpoints 
	=> in the network nothing changes 
	
	
*Pros:
 	=> Abstract and intro 
	=> Uses cases 
	=> Explained NFV pretty well

*Cons:
	=> Name drop and walk away 
	=> no good analysis 
	=> no proof of concept of the paper 
	=> Related works 

\end{comment}


































