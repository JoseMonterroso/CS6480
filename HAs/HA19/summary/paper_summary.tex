\documentclass[letterpaper,twocolumn,10pt]{article}
\usepackage{epsfig,xspace,url}
\usepackage{authblk}


\title{CS 6480: Paper reading summary\\
HA 19.a\\}
\author{Jose Monterroso}
\affil{School of Computing, University of Utah}
\begin{document}
\maketitle
\section{FALCON: An Accurate Real-Time Monitor for Client-Based Mobile Network Data Analytics}

Paper discussed in this summary is ``FALCON: An Accurate Real-Time Monitor for Client-Based Mobile Network Data Analytics''~\cite{falcon}.

\subsection{First pass information}
\label{sec:first}
\begin{enumerate}

\item {\it Category:}
This paper is a description of a research prototype. The research prototype being discussed is FALCON. 

\item {\it Context:} 
The technical area of this paper relates to mobile networking. Specifically, they are dealing with LTE and LTE-A.
This paper relates to a few papers on LTE that we have read in the past. They discuss similar concepts like RNTIs
in the security LTE~\cite{breakinglte} paper. The FALCON paper also references a recent paper on the BurstTracker
~\cite{detectinglte}. Finally, I believe that FALCON can be applied to and lives in the same realm of 5G~\cite{5gwhite}.

\item {\it Assumptions:}  
They make a few assumptions relating to how OWL is currently the most reliable open-source instrument for
continuous long-term monitoring. I've never worked with OWL but I think their assumption is valid as they 
have made FALCON and appear to be experts on the subject. 

\item {\it Contributions:}
I couldn't find and directly stated of claimed contributions. However, they present FALCON which is 
a fast analysis of LTE control channels. FALCON is also completely open source. 

\item {\it Clarity:} 
From what I have read this paper appears to be written well. However, this paper may be of shorter
length than most, but it does contain a lot of jargon.

\end{enumerate}
\subsection{Second pass information}
\label{sec:second}
\begin{itemize}

\item {\it Summary:}
In this paper the authors present FALCON: Fast Analysis of LTE Control channels. FALCON is an 
improved combination of existing open source softwares. FALCON is suitable for both long-term 
and short-term monitoring of LTE resource allocation in non-ideal radio conditions. We first go into
a little overview of control channel analysis. In this section we discuss PDCCH and DCI. We then 
touch on a few challenges in DCI validation. A few approaches to the challenges include: signal
power detection, re-encoding, RAR tracking, RNTI histograms, search space coherence and 
shortcut decoding. FALCON provides a decoder, visualization tool, signal recorder and a remote 
controller for synchronized capturing of multiple cells or mobile network operators. The signal 
recorder used by FALCON is an extended version of OWL's recorder. FALCON's decoder is 
capable of tracking either an online LTE signal or an offline recording and reliably decodes any 
resource assignments from the cell's PDCCH in real time. We then slightly touch on the privacy 
concerns of FALCON, but fear not, the payload within the allocated resources is encrypted and 
cannot be decrypted without knowledge of the secret keys. Next, the authors evaluate FALCON.
In the case of poor radio conditions FALCON outperforms OWL in average by three orders of 
magnitude. Furthermore, FALCON does not filter out any meaningful DCI for RNTI in the value 
range beyond the peak region. Lastly, they provide us with an application example of FALCON.

\end{itemize}

\subsection{Third pass information}
\label{sec:third}
\begin{itemize}

\item {\it Strengths:}
The abstract did a good job on introducing me to the problem. I was hoping to get more from the introduction but 
it didn't really work out. I think it's a huge strength that they provided FALCON as an open source tool. I thought
it was a strength that they compared their tool with OWL (the best at the time).

\item {\it Weaknesses:} 
I hated the mix of the introduction with the related works section. The related works section took over the the whole
section leaving a little paragraph briefly introducing FALCON. I wish they had used more figures to explain the 
control channel analysis. The applicability in 5G section did not answer any questions whatsoever if FALCON 
could be used in 5G. I wish they had provided more to their evaluation section. Not to be mean but overall this 
paper was weak. I felt that I didn't learn much besides what FALCON is and what it can kind of do. 

\item {\it Questions:}
I'm not familiar with LTE so I would like to know the difference between LTE and LTE-A.

\item {\it Interesting citations:} 
It was because of my current final project on radio frequency propagation that I found the paper on 
spectral efficiency ~\cite{spectralefficiency} to be an interesting read. 

\item {\it Possible improvements:} 
 I felt a bit lost when reading the paper and felt that they can improve by adding a background section
 that can go more in depth with figures and examples on the control channel. More figures in general 
 would vastly improve this paper. 

\item {\it Future work:}
I can totally see a future assignment where you can gather resource data using FALCON. They seem really
uncertain in their response to if FALCON can be used in 5G so we could test it out. 


\end{itemize}

{
  \small 
  \bibliographystyle{acm}
  \bibliography{biblio}
}
\end{document}










