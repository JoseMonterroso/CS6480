\documentclass[letterpaper,twocolumn,10pt]{article}
\usepackage{epsfig,xspace,url}
\usepackage{authblk}


\title{CS 6480: Class discussion summary\\
HA 1.b\\}
\author{Your Name}
\affil{School of Computing, University of Utah}

\begin{document}

\maketitle
\section*{Discussion summary}

\begin{itemize}

\item {\it Summary:} Summarize the class discussion of the papers. Note in
particular anything that came out of the discussion that you missed in your
reading, or that made you change your mind. 

\item {\it Strengths and weaknesses:} Note the consensus
in the class, or your own new insights, regarding the strengths and
weaknesses of the paper(s).

\item {\it Connection with other work:} Describe any connections that were made in
the class discussion to other papers (with citations). (Or connections that you realized
as a result of the class discussion.)

\item {\it Future work:} Briefly describe any possibilities
for future work that came out of the discussion and/or were triggered
in your mind because of the discussion.

\end{itemize}
{
  \small 
  \bibliographystyle{acm}
  \bibliography{biblio}
}

\end{document}




\begin{comment}

IEEE global mobile communication is second tier for network systems

Introduction and related works mushed together was very weird 
The sections were really not specific and kind mushed together

Motivation was too highlevel. It needs to be more specific or more obvious

Network data analytics function => reminded us of ORAN

Challenge is we need to make sure the RNTI value of the current active UE is valid 
	=> if not you get weird value
	
In the inrodution there wasnt an explicit section for key contributions,
	=> they don't mention how they solved certtain thiings 
	

Fig 3 the UE is trying to fiind its RNTI 
the problem for falconn is thatt it is trying to find the correct/active RNTIs




Pros 
Considered 5G enviroment 
open source
Their tools are usefull good 

Cons
Related works mushed with Introduction
Trouble keeping track of the paperr
add some figures to keep track of the process 




\end{comment}

































































