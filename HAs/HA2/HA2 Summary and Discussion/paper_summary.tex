\documentclass[letterpaper,twocolumn,10pt]{article}
\usepackage{epsfig,xspace,url}
\usepackage{authblk}


\title{CS 6480: Paper reading summary\\
HA 2.a\\}
\author{José Monterroso}
\affil{School of Computing, University of Utah}

\begin{document}

\maketitle

\section{Title of paper: Containerization and the PaaS Cloud}

Paper discussed in this summary is "Containerization and the PaaS Cloud''~\cite{containerization}.

\subsection{First pass information}
\label{sec:first}

\begin{enumerate}

\item {\it Category:} What type of paper is this? 

This paper is an analysis of containerization and how this type of virtualization in cloud computing can advance 
PaaS clouds. 

\item {\it Context:} In what technical area is the work described in the paper in?

This papers technical area relates to computer networks, specifically cloud computings ability to provide PaaS and IaaS through
virtualization. 
This type of work relates to our previous paper, "A view of Cloud Computing"~\cite{aview}, that gave as an overview of cloud
computing. This paper dives into the virtualization aspect that our previous paper failed to have. 

\item {\it Assumptions:}  What assumptions do the authors make? Do the assumptions appear to be valid?

Truthfully I couldn't find any assumptions made by the author in the introduction nor in the conclusion. From what I have read in 
the first pass this paper is aimed at showing us the benefits of containers for virtualization in cloud computing. 

\item {\it Contributions:} What are the paper's main contributions?

The author claims to investigate the relevance of the new container technology for PaaS clouds, he then will discuss
the resulting requirements for application packaging and interoperable orchestration on clusters of containers. Furthermore, the author
aims to clarify how containers can change the PaaS cloud as a virtualization technique. Lastly, he claims to discuss the achievements
and limitations of the state of the art technology.

\item {\it Clarity:} Is the paper well written?

Yes, this paper looks to be written well. He follows a nice technical format, and contains colorful figures. 

\end{enumerate}

\subsection{Second pass information}
\label{sec:second}

\begin{itemize}

\item {\it Summary:} 

In the paper, Mr. Pahl addresses the use of containerization over the current use of VMs. Throughout the paper
he provides scenarios where containerization such as Docker would be at a greater advantage than VMs. His approach
revolves around educating the reader about containerization, container-based clusters, and orchestration. The author 
follows a simple 4 section format. In virtualization and the need for containerization, the author provides a rough history 
of why virtualization technologies were created. In containerization for lightweight virtualization and application packaging, 
he discusses the LXC, and how containers are able to function as well as explaining the container-based cluster architecture.
We then finally get a formal definition of PaaS and how containers can operate in PaaS clouds, in the containerization in 
PaaS clouds section. Lastly, in the container orchestration and clustering section, we get a full understanding of clusters and how
frameworks like TOSCA can make it possible to orchestrate complex container-based topologies. 

\end{itemize}

\subsection{Third pass information}
\label{sec:third}

\begin{itemize}

\item {\it Strengths:} What are the strengths of the paper? 

I believe this paper does a good job of introducing us to a variety of virtualization technologies. Truthfully I think this paper
dives deep in the ideas of containerization and orchestration. I like how he provides multiple examples of containerization
companies and technologies as well as orchestration technologies. But by far the figures are extremely helpful in 
understanding the concepts. 

\item {\it Weaknesses:} What are the weaknesses of the paper? 

He could have done a better job of formatting his paper, ideally I would have liked to have seen more sections throughout
the paper. I also noticed that previous statements were also repeated in parts of his paper, it almost felt like a copy paste. 
One thing that really bothered me was how chose to define PaaS 5 pages into his 8 page paper, especially when he talks 
about PaaS in the preceding pages. Of the two primary issues he listed (network and data, orchestration) he didn't offer 
much of a solution. It almost felt as if he was quickly going over them so people wouldn't notice them. Lastly, the conclusion 
was really short and didn't summarize any of the main talking points. 

\item {\it Questions:} What questions do you have?

It seems as if all PaaS clouds are pushing towards using Docker, is this because Docker was one of the first to do containerization? 
I'm not at all familiar with OSs and didn't understand a lot of the linux kernel boot up arguments or host/guest architecture comparisons. 

\item {\it Interesting citations:} References that interested you.

Docker~\cite{docker} seems to be growing, so I would like to read more about it. 

\item {\it Possible improvements:} Improvements?

He simply mentions
"better networking features" when discussing a potential network and data challenge solution. So I think if he addressed that 
in more depth, as well as the challenges of containerization, his paper would improve. Furthermore, the paper would have been
better if he provided a proper conclusion to summarize his main points. 


\item {\it Future work:} Any new ideas for future work that was inspired by
your reading of the paper.

He lists three features in table 1 that compares VMs and containers. I can see future work in potentially listing even more 
features, describing pros and cons of each, and providing metics to prove said statements. I can also see future work in 
providing advanced network features to help the Kubernetes architecture. Lastly, I can see container orchestration being an
important field that a lot of people would like to research.   

\end{itemize}

{
  \small 
  \bibliographystyle{acm}
  \bibliography{biblio}
}
\end{document}












