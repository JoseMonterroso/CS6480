\documentclass[letterpaper,twocolumn,10pt]{article}
\usepackage{epsfig,xspace,url}
\usepackage{authblk}


\title{CS 6480: Class discussion summary\\
HA 2.b\\}
\author{José Monterroso}
\affil{School of Computing, University of Utah}

\begin{document}

\maketitle
\section*{Discussion summary}

\begin{itemize}

\item {\it Summary:}
We first discussed the definitions of containers and VMs. We defined containerization as a lightweight distribution
of packaged applications for deployment and management. While VMs are about hardware allocation and
management. Next we touched on the ideas of dependencies and life cycles of containers, this led to our 
primary discussion on orchestration. An orchestration plan describes components, their dependencies and container life 
cycles. Orchestration is needed for complex containers. Two orchestration 
mechanisms listed in the paper are Mesos, and Kubernetes. We then discussed the Linux kernel and how it uses
namespaces and control groups. Next we discussed how a container engine runs container
images. These image are the realization of rules that containers need to follow. 
One thing I failed to notice was the differences between micro-services and containers. Micro-services is the
approach of developing a single application as a suit of small services; each running its own process. While 
containers are the underlying technology that enables micro-services. 

\item {\it Strengths and weaknesses:} 
The consensus regarding the paper during the discussion was split. However, it seems that the weaknesses
outweigh the strengths. For example, we all seemed to agree that the paper contained great content with 
regards to the overall details covered and the amount of contented that was actually presented in the paper.
The weaknesses of the paper lie within the paper format structure, the lack of figure and table explanations,
and excessive repetition of recently presented ideas. My insights were actually identical to this as I brought 
up these same points in my paper summary.

\item {\it Connection with other work:} 
From "A View of Cloud Computing," we were introduced to the idea of virtualization. To note, 
virtualization is what allows cloud computing to work. However, that paper hardly, if any goes into specifics
about virtualization in cloud computing. From reading HA2's paper we learn about containers and VMs, which
are both types of virtualization. A small connection was that we brought up the idea of containers having
their own IP address and acting as linux routers. This in fact, is what we will be implementing in our Lab 1
assignment.  

\item {\it Future work:} 
We discussed the idea of added complexity when we use multiple apps within a container. This might
be good future work because we could analyze time constraints, connectivity constraints, and provide
a good orchestration solution to certain types of container complexities. Another Idea is the study of 
Mesos and Kubernetes orchestration in various complex driven containers. 

\end{itemize}
\end{document}

\begin{comment}
*IEEE magazine that no longer exists
	=> pretty successful but is no longer sponsored by IEEE
	
*Consensus was that this paper was not well organized 
	=> repetition throughout the article 
	
*Paper had good content though  

*Definitions of VMs versus containers 
	=> Containers are its own separate process 
	=> VM have to boot up 

*They mention papers but do not explain it well
	=> diagram was not explained well 
	=> application tier was not explained well either 
	
*Dependencies, and life cycles about containers
	=> started and stoping 

*Orchestration is needed for containerization in more complicated things 
	=> controlling the lifecycle of 

*Linux
	Namespaces: the kernel needs to have the ability to create this, different networking stack for each namespace
	Control groups: enforcement and accountant that goes on. Resources use 
	
Container image is the realization of rules that it needs to follow 
	=> file system union mount several file systems 
	=> final layer is the writable container 
	
Container engine runs the images to start the container 

*Containers can run multiple application 
	=> doesn't have to run just one
	
In clusters we need to deal with data and networking
	=> so we have links connected to a persistent state  (data volume)
	=> network port mappings and container linking 
	
*Companies using proprietary stuff and then move to open source 
	=> open source is complicated because they can have different goals and packages 
	
*Microservices vs container
	=> MS: approach to developing a single application as a suite of small services; each running its own process
		Application architecture, From Monolithic to multiple micro-services
		
	=> containers: is the underlying technology that enables micro-services 
	
*Orchestration
	=>  Mesos: runs a different kernel in your own cluster 
	=> kubernetes: higher level; yoou run processes on the docker host
	
*You can associate each container by adding one IP address for each container 
	=> complicated/ complex 
\end{comment}