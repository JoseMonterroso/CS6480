\documentclass[letterpaper,twocolumn,10pt]{article}
\usepackage{epsfig,xspace,url}
\usepackage{authblk}


\title{CS 6480: Paper reading summary\\
HA 16.a\\}
\author{José Monterroso}
\affil{School of Computing, University of Utah}

\begin{document}

\maketitle
\section{A First Look at Commercial 5G Performance on Smartphones}

Paper discussed in this summary is ``A First Look at Commercial 5G Performance on Smartphones''~\cite{commercial5g}.

\subsection{First pass information}
\label{sec:first}
\begin{enumerate}

\item {\it Category:} 
This paper is an analysis of an existing system. The analysis is on commercial 5G using commercial smartphones.

\item {\it Context:} 
The technical area of this work relates to 5G and its use of mmWave technology, and mid-band 5G operating at 
2.5 GHz. Our 5G paper~\cite{5gwhite} discussed similar topics such as 5G architecture, commercialization, and 
benefits and drawbacks of 5G. A similar measurement study ~\cite{measurementstudy} on 5G was conducted but 
rather than multiple cities, a single campus was chosen.

\item {\it Assumptions:}  
I didn't catch any assumptions during my first pass of the paper. 

\item {\it Contributions:} 
The author's contributions are as follow: they develop practical and sound measurement methodologies for 5G 
networks on COTS smartphones, they present timely measurement findings of mmWave5G and mid-band 5G 
performance on smartphones with key insight, and lastly they release their measurement dataset (5Gophers) 
to the research community. 

\item {\it Clarity:} From what I have read during the first pass, this paper appears to be well written. 

\end{enumerate}
\subsection{Second pass information}
\label{sec:second}
\begin{itemize}

\item {\it Summary:} 
The authors of this paper conduct a measurement study on commercial 5G. They study three major 5G carriers: 
Verizon, T-Mobile, and Sprint. Sprint uses mid-band 5G operating at 2.5 GHz while Verizon and T-Mobile use 
mmWave technology. This study can be boiled down too few main points. First, an overview of 
today's 5G performance. MmWave 5G throughput significantly outperforms mid-band 5G, however because of 
5G's current NSA, 5G offers little latency improvement. Next they investigate 5G performance on Stationary UE,
They find that commercial 5G offers much higher throughput than 4G. They also investigate mobility performance
of mmWave on a moving UE. They discover that 4G-5G handoffs can be triggered frequently by either network
conditions or user traffic. Mid-band 5G offers better performance due to its omni-directional radio. Next they
discover that mmWave exhibits a statistically higher throughput variation compared to 4G, due to mmWave's
sensitivity to the environment. Lastly, they study application performance over mmWave 5G and discover that
for web browsing today's 5G brings benefits only for large web pages. For HTTPS download, goodput is 
significantly lower than the available mmWave 5G bandwidth. To conclude, mmWave 5G's high bandwidth 
does not always translate to a better application QoE. 

\end{itemize}

\subsection{Third pass information}
\label{sec:third}
\begin{itemize}

\item {\it Strengths:} 
The introduction did a good job of introducing what they will be studying in commercial 5G. Regarding
the introduction section, I really thought it was a plus to have a brief overview of their findings in the later half of the 
section. Although I do not agree with the specifics of their measurement methodology I think they did a good job
of explaining what they did and why effectively. Their use of box and whisker plots make reading and understanding
their graphs better. I really thought their section 5 on how they set up their experiment and why they believe they 
got the results they did was awesome. A potential strength of this paper is that the authors do a little bit of every type
of measurement for UE's in 5G.

\item {\it Weaknesses:}
Their conclusion was very weak, it read like a copy paste of the introduction's contribution section. I can see why
they coupled their background section with their related works. They claim they are the first commercial 5G 
measurement study, however, the types of things they are measuring and the way they are measuring are not 
new things, and therefore, they could have found or sought to do similar measurement techniques from 4G which 
could have offered them more related works. I know most phones do not have 5G radios but I believe they should 
have used more than one type of phone for all their studies, because at that point it becomes a study of the cellphone
on 5G rather than on 5G itself. I'd also like to point out that a lot of information was being restated in the introduction,
section 4, section 5, and the conclusion.

\item {\it Questions:}
I find it a bit funny when they mention 2019 marks the year for 5G which was eventually rolled out for commercial
services to consumers, given that most of Asia has had 5G for a few years now. I'm still not sure what the authors
meant by saying ``due to ultra-high bandwidth of 5G the bottleneck of an end-to-end path may potentially shift from
the wireless hop of the internet."

\item {\it Interesting citations:} 
I was unaware of the creation and rise of HTTP3 so when they mentioned HTTP3~\cite{http3} during this paper
I had to go out and check it out. I remember doing my first network assignment that related to HTTP and how
familiar I had to become with the rules, so I am intrigued about the new rules and what HTTP3 can bring to the
internet. 

\item {\it Possible improvements:} 
They can improve their conclusion by providing more information about various key points throughout the 
paper. 



\item {\it Future work:} 
I think a similar study can occur when 5G grows to using the standalone deployment method. Furthermore,
this type of study can expand to other UE's and other cities. 

\end{itemize}

{
  \small 
  \bibliographystyle{acm}
  \bibliography{biblio}
}
\end{document}



















