\documentclass[letterpaper,twocolumn,10pt]{article}
\usepackage{epsfig,xspace,url}
\usepackage{authblk}


\title{CS 6480: Paper reading summary\\
HA 12.a\\}
\author{José Monterroso}
\affil{School of Computing, University of Utah}

\begin{document}

\maketitle
\section{Title of paper: A Brief Introduction to Named Data Networking}

Paper discussed in this summary is ``A Brief Introduction to Named Data Networking''~\cite{ndn}.

\subsection{First pass information}
\label{sec:first}

\begin{enumerate}

\item {\it Category:} 

This paper is a description of a research prototype. The prototype being discussed is Named Data Networking (NDN). 

\item {\it Context:} 

The technical area that relates to this paper is network architecture. In fact some aspects of the DARPAnet~\cite{darpa},
are recited within this paper. Idealy NDN would fix some of the problems associated with IP.

\item {\it Assumptions:}  

A small assumption is that the network needs to evolve along side the applications that are being developed so that 
the network can meet their demands. I believe that this is a valid assumption because we need to be able to support 
new technologies, however, we shouldn't update the network for specific use cases in technologies. 

\item {\it Contributions:} 

The authors contribute by providing a paper that offers the readers an overall picture of NDN's basic principles, concepts,
operations, and properties. 

\item {\it Clarity:} 

From what I have read, this paper appears to be well written.

\end{enumerate}
\subsection{Second pass information}
\label{sec:second}
\begin{itemize}
\item {\it Summary:} 

In this paper we learned a bit of the overall picture of Named Data Networking (NDN). NDN is designed to network 
the world of computing devices, ranging from IoT sensors to cloud servers, by naming bits rather than location. We 
can say that NDN is similar to HTTP's semantics of request and response. For NDN we use interest packets which
contains the name of the requested data, the NDA network then responds with data packets neither of which contains an
address nor information about the requestor. In NDN, consumers fetch data instead of senders pushing packets to 
destinations. Furthermore, NDN uses a stateful forwarding plane. Section four, touches on the idea of how NDN's
forwarding engine works. Next in section 5 we learn about NDN's communication security. We then learn of NDN's
need for dataset synchronization and how Sync is used to achieve this. Lastly, in section 7 we touch on the idea
of what makes NDA great for the battlefield, and how it shares various similarities and improvements from the 
TCP/IP architecture. 

\end{itemize}

\subsection{Third pass information}
\label{sec:third}
\begin{itemize}

\item {\it Strengths:} 

I believe this paper does a great job of covering the main points of NDN, as well as providing multiple further 
resources if we choose to dive in a little deeper into the concepts of NDN. I also enjoyed this paper's figures
because they enhanced the written text. Lastly, the abstract does a great job of really putting the papers work
into one solid paragraph.

\item {\it Weaknesses:} 

I feel like this paper contained more weaknesses than strengths. It's not to say that this paper was bad I just
simply think that It could have been improved. The main issues that I found relates to the format and structure
of the paper. There were a lot of places where I think moving a section before another section would have been
better. I also don't like how they relay on preexisting technologies to explain NDN concepts. And lastly, I wish
they would have provided more examples and use cases to give me more reasons why NDN is needed/better. 

\item {\it Questions:} 

I actually have a ton of questions regarding security in NDN. I don't quite understand why NDN would be 
more secure than IP. I especially don't understand when they say that NDN has natively built in security. 

\item {\it Interesting citations:} 

In the questions section above I listed my confusion with NDN security. It was also mentioned in the paper
that NDN proves to be most resilient against DoS attacks. They didn't mention why so I'm hoping that 
this paper on DoS and NDN~\cite{ndndos} will help answer some of these questions. 

\item {\it Possible improvements:} 

I believe that if they added more examples and use cases it would have made their paper stronger.  I also 
strongly believe that If they used more subsections and better name schemes their paper would be easier
to read and follow.

\item {\it Future work:} 

I would like to do a measurement study on round trip time for NDN's datagram delivery. I would also like 
to look more into NDN's self-learning function. Not much was said about the self-learning function that I 
wonder if it uses AI or machine learning to work. Another Idea could be a case study on how ineffective 
DoS attacks are against NDN compared to IP. 

\end{itemize}

{
  \small 
  \bibliographystyle{acm}
  \bibliography{biblio}
}
\end{document}























