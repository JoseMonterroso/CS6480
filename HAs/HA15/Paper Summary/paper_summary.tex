\documentclass[letterpaper,twocolumn,10pt]{article}
\usepackage{epsfig,xspace,url}
\usepackage{authblk}


\title{CS 6480: Paper reading summary\\
HA 15.a\\}
\author{José Monterroso}
\affil{School of Computing, University of Utah}

\begin{document}

\maketitle
\section{FlexRAN: A Flexible and Programmable platform for Software-Defined Radio Access Networks}

Paper discussed in this summary is ``FlexRAN: A Flexible and Programmable platform for Software-Defined Radio Access Networks''~\cite{flexran}.

\subsection{First pass information}
\label{sec:first}
\begin{enumerate}
\item {\it Category:}

This paper is a description of a research prototype. The research prototype being discussed is FlexRAN. FlexRAN is the 
worlds first open-source SD-RAN platform. And in this paper we dive into the implementation and mechanics of FlexRAN.

\item {\it Context:} 

The technical area of this paper relates to Radio Access Networks (RAN). In fact we also touch onto elements of NFV~\cite{nfv}
and 5G~\cite{5gwhite}. 

\item {\it Assumptions:} 

They assume that they are the first Open source RAN platform. I'm actually not familiar with RAN. so I can't say for sure
if FlexRAN is the first. Furthermore, they assume that SDN is highly needed to manage a lot of time requirements for 5G. 
I think the assumption is true as far as saying that SDN can add more to RAN but it is still up to the user to define how 
SDN can be used to improve RAN.

\item {\it Contributions:} 

The authors of this paper have a bulleted list of contributions. First, they contribute by creating FlexRAN which incorporates an
API for clean separation of control and data planes. They also contribute by implementing FlexRAN over the OpenAirInterface 
LTE platform. And lastly, they show results from using FlexRAN in a diverse set of use cases relevant to current and future mobile
networks. 

\item {\it Clarity:} 

From what I have read this paper is written well and appears to be professional.

\end{enumerate}

\subsection{Second pass information}
\label{sec:second}

\begin{itemize}
\item {\it Summary:}

FlexRAN at the time of its creation was the world's first open-source SD-RAN. FlexRAN incorporates an API to 
separate the control and data planes. It offers programmability at two levels, one in the form of RAN control and 
management applications that can be built over the FlexRAN controller and the other within the controller to be 
able to upgrade the implementation of any control function in real time. Furthermore, FlexRAN is transparent to
UEs. For FlexRAN, eNodeBs only handle the data plane to perform all the action-related functions. The northbound
API, allows RAN control and management applications to modify the state of the network infrastructure. While the 
southbound API is the primary enabler for SDN control of the RAN. To allow flexible and programmable control of the
RAN, the FlexRAN agent provides multiple eNodeB control modules. Each module controls subsystems targeting a
specific area of control. Without going to much into detail FlexRAN uses more CPU utilization and memory than that
compared to OAI. However, both FlexRAN and OAI experienced the same service quality. Lastly, FlexRAN proves to 
a useful tool for the SD-RAN platform. 

\end{itemize}

\subsection{Third pass information}
\label{sec:third}
\begin{itemize}

\item {\it Strengths:}

I liked the abstract and introduction because they do good job of explaining RAN and why most SD-RAN platforms
out there do not really help or are physically built to make RAN better. The overall format of this paper was well done. I especially 
enjoyed the little overview section at the beginning that prepped me for the in-depth FlexRAN. Another strength was 
the system evaluation section, I thought it was very thoughtful they they included the specs of machines used for their
evaluation. FlexRAN use cases bring up an interesting and applicable solutions for SD-RAN. I think it's a great idea for 
them to take into account what SD-RAN could do, however, this might not be the right paper for this.

\item {\it Weaknesses:} 

Although the introduction was a nicely written section, I felt that their contributions at the end of the section were weak.
They didn't really feel like contributions, in fact it felt like they were simply describing the future sections of the paper. 
I think it's a bit of a stretch to envision an online VSF store similar to mobile app stores, given that VSF are an inherently 
important aspect of the FlexRAN system. Another thing I didn't like in this paper was the Extending OpenAirInterface with 
FlexRAN section, it didn't really contribute to the paper, but rather felt like a self-praise by the authors who had to write 10,000
lines of code to integrate FlexRAN into OAI. I do have to say that their evaluation section was a bit weak, and short compared to
the build up of introducing FlexRAN and the design behind FlexRAN. 

\item {\it Questions:} 

I'm not sure what pub/sub communication is. I also have questions as to what the RRC, MAC/RLS,  and PDCP protocols are
in LTE. I'm also not familiar with heterogenous networks. 

\item {\it Interesting citations:} 

The 5G white paper in class was good, but relied heavily on previous 4G technologies to explain 5G components and 
differences. Because of this I feel that I do not have a good foundation or even good knowledge of 5G, that is why I am
picking the NGMN~\cite{ngmn} white paper on 5G. 

\item {\it Possible improvements:} 

A few statements were constantly repeated throughout the paper. I believe removing these repeated statements will 
improve the paper. I wish they cut down their introduction sections and went more in depth for the evaluation section. 

\item {\it Future work:} 

I think this can definitely extend to 5G NSA or standalone infrastructures. I would also like to expand the master control
to work on non-linux systems. Furthermore, FlexRAN is not tied down to specific RAN so the possibilities are endless. Lastly,
because this is new technologies there are a variety of places or things this technology could apply to and improve.

\end{itemize}

{
  \small 
  \bibliographystyle{acm}
  \bibliography{biblio}
}
\end{document}



