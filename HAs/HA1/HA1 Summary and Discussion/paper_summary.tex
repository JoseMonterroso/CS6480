\documentclass[letterpaper,twocolumn,10pt]{article}
\usepackage{epsfig,xspace,url}
\usepackage{authblk}


\title{CS 6480: Paper reading summary\\
HA 1.a\\}
\author{José Monterroso}
\affil{School of Computing, University of Utah}

\begin{document}

\maketitle

\section{A View of Cloud Computing }

Paper discussed in this summary is "A View of Cloud Computing''~\cite{aview}.

%%%%%%%%%%%%%%%%%%%%%%%%%%%%%%%%%%%%%%%%%%%%%%%%%
%%% First Pass
\subsection{First pass information}
\label{sec:first}

\begin{enumerate}

\item {\it Category:} What type of paper is this?

The type of this paper is an analysis of an existing system because the authors of the paper explain what cloud computing is and discuss the top ten obstacles and opportunities of cloud computing. 


\item {\it Context:} In what technical area is the work described in the paper in? What other papers does it relate to? 

The technical area of this work is computer networking, specifically cloud computing use of the internet to provide the user services.
I've read papers about SaaS, and IaaS in my upper division writing class which relate to cloud computing. 

\item {\it Assumptions:}  What assumptions do the authors make? Do the assumptions appear to be valid?

The authors assume the readers have background knowledge of the IT industries need for cloud computing. I think this assumption is a bit invalid because the purpose of the article is to explain what cloud computing is and its top pros and cons. Furthermore, they assume cloud computing is a hot topic. Now this article was written 10 years ago, but for 2020 I believe machine learning is the hot new thing. 

\item {\it Contributions:} What are the paper's main contributions?

The contributions of this paper revolve around providing a detailed explanation of what cloud computing is, the types of cloud computing, its economic costs, and a set of pros and cons. 

\item {\it Clarity:} Is the paper well written?

From what I have read in the introduction and the conclusion, I do believe this paper is well written because it clear explains what this paper is about and why it should be read. 

\end{enumerate}

%%%%%%%%%%%%%%%%%%%%%%%%%%%%%%%%%%%%%%%%%%%%%%%%%
%%% Second Pass 
\subsection{Second pass information}
\label{sec:second}

\begin{itemize}

\item {\it Summary:} Summarize the paper in your own words. 

The problem being addressed in this article is to reduce the confusion of cloud computing by clarifying 
terms, providing simple figures to quantify comparisons between the cloud and conventional computing,
and identifying the top technical and non-technical obstacles and opportunities of cloud computing~\cite{aview}.
The article starts by defining what cloud computing is. They describe it as applications delivered as services
over the internet while the hardware and systems software is located at data centers~\cite{aview}. We then learn about
three classes of utility computing through the use of example clouds. A low level class being Amazon EC2, 
a mid-tier level being Microsoft Azure, and a high level being Google AppEngine. Next we are introduced to the 
economic costs of cloud computing. Ideally how cloud computing in a "pay as you go" model offers economic benefits. 
Lastly, before the conclusion 
we come across the top 10 obstacles and opportunities for growth of cloud computing. These 10 obstacles and
opportunities relate to the clouds ability to handle data transfers, security, performance, scalability, data storage,
and bugs in the code. Finally, the conclusion tells us that the authors of the paper predict that cloud computing 
will grow. Furthermore, they provide readers with 3 primary takeaways of a successful cloud computing system. 

\end{itemize}

%%%%%%%%%%%%%%%%%%%%%%%%%%%%%%%%%%%%%%%%%%%%%%%%%
%%% Third Pass
\subsection{Third pass information}
\label{sec:third}

In this section we move from factual aspects of the paper to
your opinion about the paper.

\begin{itemize}

\item {\it Strengths:} What are the strengths of the paper? 

I believe the use of figures, tables, and images made this paper more lively and susceptible to reading. 
The introduction was a strength because it clearly introduces the concept of cloud computing, it also slightly 
explains why cloud computing is important, therefore, legitimizing the need for this paper, and lastly outlining
what will be discussed in the paper. Another strength is Table 2. The meat of this paper revolves around the 
idea of the top 10 obstacles and opportunities of cloud computing. But thanks to Table 2, this information is presented
upfront. There's nothing worse than reading a paper and not being able to find the answers that the paper claims to solve/explain.

\item {\it Weaknesses:} What are the weaknesses of the paper? 

I honestly hated the ACM advertisement on page 56 because it takes me away from the reading. But aside from that
I found it really difficult to find weaknesses in the paper. I believe I don't know enough about cloud computing to find
contradictions or hold different point of views against the paper. 

\item {\it Questions:} What questions do you have about the paper? 

I'm confused about the concept of virtualization with regards to cloud computing and how it is used/helpful.
I also have questions regarding the differences between Amazon EC2, Azure, and Google AppEngine. They
all seem to provide the same things, but are the differences mainly computing power and scalability? 

\item {\it Interesting citations:} References that interested you.

For me, the most interesting reference I found was the the FBI raid on a Dallas computer center ~\cite{fbi}.
This raid cost user services to go down. It would be interesting to see if any users sued the FBI, and if those
users lost their data.

\item {\it Possible improvements:} Any ways in which the work described in
the paper could be improved? 

I honestly though that this paper was really good. But it almost seems a bit biased towards pushing you into
cloud computing. I think it would be nice if they offered obstacles that don't nessasarily have a good solution or 
opportunities of growth. I think this would help provide readers a more realistic view of cloud computing. 

\item {\it Future work:} Any new ideas for future work that was inspired by
your reading of the paper.

I for one am not too familiar with the insides of cloud computing, so it would be nice to look into the behind the 
scenes aspect of cloud computing. In the paper they discuss the economics of cloud computing, I think a potentially 
good future paper could revolve around the idea of the "pay as you go" model, and the benefits and disadvantages of 
over-provisioning and under-provisioning resources. 

\end{itemize}

{
  %\footnotesize 
  \small 
  \bibliographystyle{acm}
  \bibliography{biblio}
}
\end{document}






















