\documentclass[letterpaper,twocolumn,10pt]{article}
\usepackage{epsfig,xspace,url}
\usepackage{authblk}


\title{CS 6480: Class discussion summary\\
HA 1.b\\}
\author{José Monterroso}
\affil{School of Computing, University of Utah}

\begin{document}

\maketitle
\section*{Discussion summary}

\begin{itemize}

\item {\it Summary:} 
During the discussion in class we first defined what cloud computing is. Now something that I missed was that 
the authors of the paper define cloud computing as applications delivered as services over the internet while the 
hardware and systems software is located at the data centers. But they then discuss how enterprise data centers using 
virtualization is not a cloud. So we see the authors contradict themselves a bit. In the discuss we then moved on to
the classes of utility computing. What I missed in the reading was noticing the difference of the model of computations used
for Amazon EC2, Azure, and AppEngine. I failed to realize the different layers of abstraction used in each cloud. For example,
EC2 is a lower level of abstraction meaning we have more control over what we would like to do in our EC2 instance. However,
AppEngine is higher level meaning that Google handles most of the backend and we need to learn how to use their APIs. 
We then moved on to discuss the 10 obstacles and opportunities of growth. I believe the most technically challenging obstacle
is scalable storage. Even the authors of the paper mention that we still need to create scalable storage.  

\item {\it Strengths and weaknesses:} 
I believe the consensus of the class was that this is a good paper. The paper aims to teach readers about cloud
computing and I think it does that without overwhelming the reader. I mentioned this before in the summary, but the 
definition of cloud computing didn't include the idea of scalability. So when they mention an enterprise database not being
a cloud, their definition seems to falter. Another weakness was the constant use of the word virtualization in a broad way.
I felt that the authors defaulted to the word "virtualization" when they discussed how the cloud runs services. Finally,
a strength of the paper was the discussion of scalability and elasticity. In my opinion this is the number one reason why I 
would use cloud computing. 

\item {\it Connection with other work:}
Because this is the first paper we discussed in class no other papers were referenced during the discussion. However, 
some connection to cloud computing that I have used revolves around Stadia, and Amazon EC2. I read previous papers
and documentation on Amazon EC2 for my web development class, were I first learned about elasticity. Secondly, stadia
is Google's cloud gaming platform.  

\item {\it Future work:} 
It was mentioned in the discussion to incorporation some kind of ML or AI to appropriately trigger cloud computing elasticity.
Another idea could be the study of the models of computation used by different cloud providers and how/if they attempt to data lock you. 

\end{itemize}

\end{document}































