\documentclass[letterpaper,twocolumn,10pt]{article}
\usepackage{epsfig,xspace,url}
\usepackage{authblk}

\title{CS 6480: Paper reading summary\\
HA 23.a\\}
\author{Jose Monterroso}
\affil{School of Computing, University of Utah}
\begin{document}
\maketitle
\section{On the Needs and Requirements Arising from Connected and Automated Driving}

Paper discussed in this summary is ``On the Needs and Requirements Arising from Connected and Automated Driving'~\cite{ontheneeds}.

\subsection{First pass information}
\label{sec:first}
\begin{enumerate}

\item {\it Category:} 
This paper is an analysis of an existing system as well as a description of a research prototype. The existing system
that is being analyzed is 5G's ability to support mission-critical Vehicle-to-Everything communications. While the research
prototype revolves around the main vehicle-to-Everything application categories and their representative use cases selected
based on the analysis of the future needs of cooperative and automated driving. 

\item {\it Context:} 
The technical area of this paper relates to mobile networking, specifically 5G~\cite{5gwhite}. And Vehicle-to-Everything (V2X)
communications. This will be our first V2X paper so not much can relate to this paper.

\item {\it Assumptions:}  
The authors make a few assumptions relating to the fact that 5G has made V2X a mission-critical goal for their network.
I think their assumptions is valid as they tell us of how, ADAS, 3GPP, ETSI, ITS, 5GAA, and 5GCAR have all been brought
together to figure out how and why V2X can be put together within 5G.

\item {\it Contributions:} 
The paper's main contributions are as follows. First, they identify representative use cases that are based on an 
analysis of the demands arising from connected and automated driving. Secondly, they study the selected use cases in 
more detail to identify the corresponding challenging requirements and derive the key performance indicators (KPIs). And 
lastly, based on the identified use cases and requirements, they discuss the existing V2X technologies and solutions and
point out valuable future research directions for satisfying the stringent requirements. 

\item {\it Clarity:}
From what I have read in the first pass this paper appears to be written well.

\end{enumerate}
\subsection{Second pass information}
\label{sec:second}
\begin{itemize}

\item {\it Summary:} 
The authors of this paper provide a description of the main V2X application categories and their representative use
cases selected based on an analysis of the future needs of cooperative and automated driving. In this paper they 
summarize and extend the key findings of the 5GCAR project in terms of use cases that form building blocks for 
connected automated driving. Section 4 proposes a methodology for deriving the requirements on the communication
systems from the automative requirements, which is then applied to specify the requirements for the selected use 
cases.The authors define the automotive and network KPIs; then, they discuss the interrelations between them.
Section 5 draws the perspective between the use case and requirements previously identified and the one defined 
by the community, principally 3GPP and the 5GAA. Section 6 reviews related work and future research challenges.
Finally, section 7 gives a conclusion to the paper.

\end{itemize}
\subsection{Third pass information}
\label{sec:third}
\begin{itemize}

\item {\it Strengths:} 
Table 2 does a great job of simplifying and combining the overall data of each use case with the needed requirements.
I like section 5 because it goes back to what the original idea makers had when envisioning V2X. We see that from 
different committees (3GPP, 5GCAR) has similar and different use cases ideas for V2X. I though that the technical
challenges posed in section 6 brought a good sense of what needs to be done before we can get good V2X
communication. Not only that but they provide a table with technical solutions that can be applied to improve the 
identified challenges. 

\item {\it Weaknesses:} 
Some of the communication network KPIs didn't really make much sense to me. Especially interpreting figure 6
was a bit difficult. A lot of the number used in the use case section with KPIs seemed a little off. They didn't really
explained where they got the number they kind of just show up and told us to consider them. 

\item {\it Questions:} I question a lot of the numbers they throw up in the section 4. Some make sense but others
seem a bit creative.

\item {\it Interesting citations:} 
After reading this paper a lot comes to mind relating to the sheer number of sensors and little gadgets within a car
that needs to be working properly at all time. Now you can image in a busy traffic area where these sensors still need
to be in tip-top shape to keep the passenger safe. Given my research project that relates to propagation models for 
radio frequencies in POWDER, I found the paper on vehicular communications~\cite{vehicularcomm} that deals with
channel and propagation models to be and interesting citation. I can see how this might be useful information for cars
in crowded areas.

\item {\it Possible improvements:} 
I honestly think that they can remove section 4 and just use table 2 to explain the ideas for each use case. Section
4 has a lots of overused data that in my opinions looks a lot nicer within the table. I would have like to have seen more 
evaluation about why the different organizations chose different use cases and how that could have an effect on V2X.

\item {\it Future work:} 
A future work idea I am interested in is studying propagation models using vehicular communications or even mobile
UEs. 



\end{itemize}

{
  \small 
  \bibliographystyle{acm}
  \bibliography{biblio}
}
\end{document}



