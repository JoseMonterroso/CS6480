\documentclass[letterpaper,twocolumn,10pt]{article}
\usepackage{epsfig,xspace,url}
\usepackage{authblk}


\title{CS 6480: Class discussion summary\\
HA 3.b\\}
\author{José Monterroso}
\affil{School of Computing, University of Utah}
\begin{document}
\maketitle
\section*{Discussion summary}
\begin{itemize}

\item {\it Summary:} 
We spent quite a bit of time discussing the abstract and the the introduction. It was brought to our
attention that these two sections are considered to be great sections to mimic if we decided to do a systems
paper of our own. What's great about these sections is that they present the problem up front and lay 
the groundwork/background for the more technical sections later on. We then moved on to talk about 
east and west traffic. This was one of the questions I had. East west traffic is a way of interpretting
how microservices interact with other microservices. And this lead to our main discussion on eZTrust.
In traditional firewall approach we write up rules for ip address and ports, but for microservices this doesn't
apply. So what eZTrust does is use eBPF's eTagger and eVerifier to add context tags to each packet 
and upon arrival it verifies the packet by checking its tag. 


\item {\it Strengths and weaknesses:} 
Towards the end of the class we spent sometime discussing some pros and cons regarding the paper. I'd 
say that overall there was mixed feelings about this paper. Of the parts we discussed it was brought up that 
the abstract and introduction were good. Another strength that was discussed were the graphs, algorithms, 
figures, and tables. Figure 1 and 2
really brought out the design ideas, and put the logic into concrete visualizations. One final strength that was brought up 
was the organization of the paper. In previous papers that we have discussed they were lacking concrete sections.
But this paper does a great job of introduction the problem and solution, explaining eZTrust's design and evaluating
its performance compared to other related works.  Now the weaknesses mentioned during our discussion related
to the contents of the paper. They present a new system for security, but rather than focusing on the security aspects
and analysing new potential flaws, they focus on evaluating their performance. It's really weird to design a security 
system and not focus on its security abilities. 

\item {\it Connection with other work:} 
During this discussing at least, there wasn't any mentions or talks of other papers. But from my knowledge this paper
talks about microservices, lightweight containers, and containerization. All of these topics relate to our HA2 paper "Containerization 
and the PaaS Cloud. " 

\item {\it Future work:} 
We brought up the ideas of eBPF and how on a packet per packet basis it can use the eTagger, and eVerifier
I think there is a lot of potential for this type of system. This can be used to trace packets, or get data regarding 
on how microservices use packets. 

\end{itemize}


\end{document}

\begin{comment}
*Abstract 
	=> good abstract, it tells you the problem up hand 
	=> micro serivce architrue opens up new attacks

*Cross service dependencies
	=> two microserrvices in the packet end 
	
*Traffic used to come in North South
	=> But now microservices that interact with other microservices is like an east west direction

*In a traditional firewall approach 
	=> you right the rules for ip address and port
	=> does not apply to microservices because they change for them 
	
We want the context of each service but we want a skinny packet
	=> so we created a tag for each mircoservice 
	
Introduction is good because it is similar to abstract where it provides rough layout with small details in each section
	=> The Introduction builds on the abstract by providing more details 
	=> pretty good model for systems paper 
	
We listed the 3 contributions
	=> pretty generic list of contributions 
	
Related work (good)
	=> good that they go through others projects and show how eZTrust can solve what they are lacking 
	
Threat model
	=> lots of assumptions 
	=> descriptions of different attack vectors 
		=> here are the boundaries of my work, you box yourself in 
	=> Needs examples 
		
Need to focus on performance 

Good strategy to give a small examples and then list the requirments 

eBPF - In the etagger and the everfier 

etagger puts the tag in the packet, egress
everifier , ingress 
	=> 4 step process
	
etracer, used to create the contrext 

harvester, interacts with microservices to collect context 

*Discussed pros and cons of paper
	=>Pros 
		abstract, introduction, evaluation, good organized, graphs, formatted well, motivational topics, diagrams 
		
	=> Cons
		some places were hard to read, and not fully fleshed
		How there paper introduces new security issues but they are not address
		doesn't really evaluate their security beneffits just performance things 
		The threat model was not good enough, could have been better points that they resolve  
		
		
\end{comment}




























