\documentclass[letterpaper,twocolumn,10pt]{article}
\usepackage{epsfig,xspace,url}
\usepackage{authblk}


\title{CS 6480: Class discussion summary\\
HA 1.b\\}
\author{Your Name}
\affil{School of Computing, University of Utah}

\begin{document}

\maketitle
\section*{Discussion summary}

\begin{itemize}

\item {\it Summary:} Summarize the class discussion of the papers. Note in
particular anything that came out of the discussion that you missed in your
reading, or that made you change your mind. 

\item {\it Strengths and weaknesses:} Note the consensus
in the class, or your own new insights, regarding the strengths and
weaknesses of the paper(s).

\item {\it Connection with other work:} Describe any connections that were made in
the class discussion to other papers (with citations). (Or connections that you realized
as a result of the class discussion.)

\item {\it Future work:} Briefly describe any possibilities
for future work that came out of the discussion and/or were triggered
in your mind because of the discussion.

\end{itemize}
{
  \small 
  \bibliographystyle{acm}
  \bibliography{biblio}
}

\end{document}



\begin{comment}

IEEE communications magazine
	=> Not that 
	
The figures were not really necessary 

The motivations was not the best 

6G paper is fine
	=> not amazing disruptive ideas but provide fair assumptions of what could be
	
6G should handle 10s of millions of connections
	=>  Micro second latency 
	=> terrabit throughputs 
	

6G
	=> innnovative network archhitecures
		=> terahertz, optical communicaitons (VLC)
	=>Innovative network archittectures
		=> cell-less architectture 
	=> Integrating intelligence in the network
		=> ML and AI into the network

Figure 2
	=> combining all these use cases and technologies together 
		=> instead of using slices


7 Nines of reliability 
	=> 99.99999\% up time 
		=> basically it never goes down

Unmanned mobility 
	=> in theory push to the cloud
		=> but didnt really buy it as it means if the network is down
		
sub-6 GHz is legacy pretty much anything less than 6 GHz

Low path-loss is good 

4G is in the figure 3 macro region

Smart city is IoT devices 

Terahhertz means really short links 


Full Duplex - like TDD use the same channel but send and receive at the same time 
	=> really hard because of self interference 
		=> sending a big signal while receiving a small signals 
		
		
tight integration of mulitple
	=> do not coonnect to a cell but to the network 
	


Future WORK IDEAS
	=> Drones as routers
		=> Flying cellsites on wheels (COWS)
	=> leveraging a 3d Network architecture for rapid network relief
	=> Measurments on VLC and terahertz
	=> bandwidth stress test for holograms 
	=> appliing machine learning and AI to troubleshoot the network and become more self aware
	=> using terahetz indoors and 4G or 5G to transmit the data 
	=> 6G deployment, cannot use the same cell towers 
	=>
	=>


\end{comment}

























































