\documentclass[letterpaper,twocolumn,10pt]{article}
\usepackage{epsfig,xspace,url}
\usepackage{authblk}


\title{CS 6480: Paper reading summary\\
HA 20.a\\}
\author{Jose Monterroso}
\affil{School of Computing, University of Utah}

\begin{document}

\maketitle
\section{Toward 6G Networks: Use Cases and Technologies}

Paper discussed in this summary is ``Toward 6G Networks: Use Cases and Technologies''~\cite{6g}.

\subsection{First pass information}
\label{sec:first}
\begin{enumerate}

\item {\it Category:} 
To me this paper is a mix of an analysis of an existing system while also being a description a research prototype.
This paper discusses use cases and topics that the authors deem to be 6G related in nature. As well as provide,
advancements from 5G that can be added to 6G mobile networking. 

\item {\it Context:}
The technical area of this paper relates to mobile networking, as well as mobile networking architecture. The paper discusses
6G technology. Some of the technologies of 6G extend from 5G ~\cite{5gwhite}.

\item {\it Assumptions:}  
There are a few assumptions that they make about 5G not being able to handle the future trends of technology. Another
assumption they make is the direction of future mobile networks. I agree with their assumptions of 5G as it has
been proven by the lack of the initial designs of 5G. However, I do not agree with their assessment of the 
fully digital world as we are not sure what the future holds and what potential decision we will have to make 
regarding situations we will have to make in the future. 

\item {\it Contributions:} 
The papers contributions are as follows. First, they adopt a systematic approach in analyzing the research 
challenges associated with 6G networks. They provide a full-stack perspective, with considerations related to 
spectrum usage, physical medium access, and network architectures and intelligence for 6G. Secondly, they 
transfer into their work a multifaceted critical spirit with solutions that they view show the highest potential for 
future 6G systems. 

\item {\it Clarity:} From what I have read in the first pass this paper feels like an easy to read, well made
paper. I also like the designs and figures they include in the paper.

\end{enumerate}
\subsection{Second pass information}
\label{sec:second}
\begin{itemize}

\item {\it Summary:} 
This paper provides a full-stack, system-level perspective on 6G scenarios and requirements, while also
selecting 6G technologies that can satisfy them either by improving 5G designs or by introducing completely
new communication paradigms. The authors of this paper envision a new approach for future mobile networks.
They discuss novel disruptive communication technologies, innovative network architectures, and being able
to integrate intelligence into the network. They then move on to discuss 6G use cases. Augmented reality and 
virtual reality will deplete the 5G spectrum, thus 6G will need to support gigabit-per-second rates to keep up.
The next use case discussed is holographic telepresence. We then touch of EHealth and how continuous 
connection availability, ultra-low latency and mobility support will be needed for this use case to succeed.
Pervasive connectivity is the notion of growing mobile devices, and how 6G will need to handle the workload.
Lastly, with regards to use cases they discuss industry and robotics, and unmanned mobility. In the next section
discussing 6G enabling technologies, the authors consider physical layer breakthroughs, new architectural and 
protocol solutions, as well as applications to artificial intelligence. Next, the authors discuss the main 
architectural innovations that 6G will introduce given by the disruptive communication technologies. To conclude,
the authors discuss integrating intelligence into the network.

\end{itemize}
\subsection{Third pass information}
\label{sec:third}
\begin{itemize}

\item {\it Strengths:} 
I really enjoyed all the figures they included. The figures really improved the quality of the paper. I like how they 
proposed the use cases first and then expanded out to the necessary technologies and the implications it will 
have on the networking architecture. I also like how they not only described how certain aspects of the 6G network
architecture will be needed but they also provided potential drawbacks and difficulties in making such technologies
work.

\item {\it Weaknesses:} 
I thought that a few of their use case could use more content. They do a good job of bringing up 5G's inability
to effectively handle the use cases but they don't really provide much more within each use case paragraph. 
However, I thought this article was well made and provided me with decent knowledge of 6G. 

\item {\it Questions:} 
They mention cell-less mobile networking paradigms. I would be curious to see how they would think to 
accomplish this. I can see VLC being very difficult to use as almost any type of interference be that of 
wind or even shadow could interfere with its connection. 

\item {\it Interesting citations:} 
I think it's really cool to see technology grow to the point where it takes care of its self. For, example self 
driving cars are a new thing with growing prospects. In today's world, cars are almost equals to that of 
the technology of the phone. That is why I am interested in learning more about vehicular communication using
millimeter-wave~\cite{vehicular}.

\item {\it Possible improvements:} 
Although I understand this paper is fairly new in release data and 6G idea space I feel that they could have done 
a bit more research and provided more specifically targeted references to each use case and enabling technology.

\item {\it Future work:} 
A few things I thought of were measurements studies of VLC and terahertz communications. Drones as routers that
can provide internet connection to people. They also mention intelligence and I can see applying machine learning or even small
AI functionalities to improve the network and also troubleshoot the network when it goes down.


\end{itemize}

{
  \small 
  \bibliographystyle{acm}
  \bibliography{biblio}
}
\end{document}










