\documentclass[letterpaper,twocolumn,10pt]{article}
\usepackage{epsfig,xspace,url}
\usepackage{authblk}
\title{CS 6480: Paper reading summary\\
HA 22.a\\}
\author{Jose Monterroso}
\affil{School of Computing, University of Utah}
\begin{document}
\maketitle
\section{On the Needs and Requirements Arising from Connected and Automated Driving}

Paper discussed in this summary is ``On the Needs and Requirements Arising from Connected and Automated Driving''~\cite{ontheneeds}.

\subsection{First pass information}
\label{sec:first}
\begin{enumerate}

\item {\it Category:} 
This paper is an analysis of an existing system as well as a description of a research prototype. The existing system
that is being analyzed is 5G's ability to support mission-critical Vehicle-to-Everything communications. While the research
prototype revolves around the main vehicle-to-Everything application categories and their representative use cases selected
based on the analysis of the future needs of cooperative and automated driving. 

\item {\it Context:} 
The technical area of this paper relates to mobile networking, specifically 5G~\cite{5gwhite}. And Vehicle-to-Everything (V2X)
communications. This will be our first V2X paper so not much can relate to this paper.

\item {\it Assumptions:}  
The authors make a few assumptions relating to the fact that 5G has made V2X a mission-critical goal for their network.
I think their assumptions is valid as they tell us of how, ADAS, 3GPP, ETSI, ITS, 5GAA, and 5GCAR have all been brought
together to figure out how and why V2X can be put together within 5G.

\item {\it Contributions:} 
The paper's main contributions are as follows. First, they identify representative use cases that are based on an 
analysis of the demands arising from connected and automated driving. Secondly, they study the selected use cases in 
more detail to identify the corresponding challenging requirements and derive the key performance indicators (KPIs). And 
lastly, based on the identified use cases and requirements, they discuss the existing V2X technologies and solutions and
point out valuable future research directions for satisfying the stringent requirements. 

\item {\it Clarity:}
From what I have read in the first pass this paper appears to be written well.

\end{enumerate}
\subsection{Second pass information}
\label{sec:second}
\begin{itemize}

\item {\it Summary:}
The authors of this paper provide a description of the main V2X application categories and their representative use
cases selected based on an analysis of the future needs of cooperative and automated driving. In this paper they 
summarize and extend the key findings of the 5GCAR project in terms of use cases that form building blocks for 
connected automated driving. Section 2 discusses the future needs that arise from connected and automated 
driving and proposes a classification of the related novel V2X applications. Specifically, they discuss the process 
of gradually allowing for the driver to remain out of the driving loop. This brings us to the sensors need to allow this to
happen. The sensors fall into two categories, line-of-sight where the car can only see what is around it, and behind the
corner which uses sensors from other cars and detailed maps to see a 360-degree environment. Section 3 describes 
five representative use case classes that have been elaborated in 5GCAR and will contribute to making the vision of
connected and automated driving become a reality. Furthermore, section 3 translates the technical requirements from
the automated domain into requirements for the telecommunication network. 

\end{itemize}

\subsection{Third pass information}
\label{sec:third}
\begin{itemize}

\item {\it Strengths:} 
I thought they they did a great job of backing up their claims on how V2X is an important topic that is growing under
the current 5G deployment. They did this by bringing up other internet, networking, and automative organizations by
discussing their contributions and ideas for the V2X space. I like how they added little examples to each use case class
it really solidified the idea. 

\item {\it Weaknesses:}
The table on page 4 was kinda pointless and listed redundant information that was described in better detail below the 
table. They mention things like the road fusion function and what it does but don't really go into any detail on it.

\item {\it Questions:} 
To accomplish theses task in sections 2, and 3 a lot of data is needed. I question if 5G will be able to handle the needs
of V2X. I'm also curious to see what companies will choose in order for a car to avoid an accident. For example, in order
to avoid an accident will the car move to the side walk or run a red light potentially endangering civilians or other cars. 

\item {\it Interesting citations:}
I find any type of autonomous vehicles/drones to be a really cool topic because I seem to always picture the future with 
robotics acting in such a way. And now with the implementation of 5G into our network a lot of these things come into 
reality. That is why I find the white paper on automotive implications on 5G~\cite{perspectiveson} to be an interesting read. 
This paper appears to tackle some interesting use cases relating to 5G from the automotive and transport and logistics 
perspectives. 

\item {\it Possible improvements:} 
I think a background section would fit this paper very well, as I was a bit confused when they said so and so will be used
to do such and such. 

\item {\it Future work:} 
Collision studies on what the car will choose to best avoid an accident. Such technologies could also be used for 
space docking and other vehicles. I can also see to achieve excellent V2X communication the network will heavily 
benefit and things that require higher throughput and really low latency will excel during this time. So you could say 
a paper on what other technologies will benefit from enabling V2X communication. 

\end{itemize}

{
  \small 
  \bibliographystyle{acm}
  \bibliography{biblio}
}
\end{document}



