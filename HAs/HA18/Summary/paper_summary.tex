\documentclass[letterpaper,twocolumn,10pt]{article}
\usepackage{epsfig,xspace,url}
\usepackage{authblk}


\title{CS 6480: Paper reading summary\\
HA 18.a\\}
\author{Jose Monterroso}
\affil{School of Computing, University of Utah}

\begin{document}

\maketitle
\section{Detecting if LTE is the Bottleneck with BurstTracker}

Paper discussed in this summary is ``Detecting if LTE is the Bottleneck with BurstTracker"~\cite{detectinglte}.

\subsection{First pass information}
\label{sec:first}

\begin{enumerate}

\item {\it Category:} 
This paper is a description of a research prototype. The research prototype being discussed is BurstTracker. 

\item {\it Context:} 
The technical area of this paper relates to mobile networking, specifically 4G LTE. Our previous paper 
on the security flaws of LTE~\cite{breakinglte} relate in that they both discuss LTE. However, the 5G 
paper~\cite{5gwhite} can also relate as it discusses similar concepts between 4G and 5G. 

\item {\it Assumptions:} 
Within the first path the authors do not make any bold assumptions. But rather focus on the usability of 
BurstTracker on video streaming performance.

\item {\it Contributions:} 
They list their contributions are as follow. First, they observe that the scheduling patterns of LTE base 
stations reveal the status of each user's downlink queue. Secondly, they describe an algorithm 
for determining when the LTE link is the bottleneck and they use it to implement a client-side LTE 
bottleneck estimator called BurstTracker. Furthermore, they demonstrate that BurstTracker can 
estimate time periods that the LTE downlink is the bottleneck. Thirdly, using BurstTracker they 
discover that many Tier-1 providers in the US appear to operate middleboxes that perform slow
start restart on HTTP and HTTPS flows. Lastly, they observed that SSR causes video streaming
applications to underutilize the LTE downlink due to a negative feedback loop in bitrate selection. 

\item {\it Clarity:} 
From what I have read this paper appears to be written well. 

\end{enumerate}
\subsection{Second pass information}
\label{sec:second}
\begin{itemize}

\item {\it Summary:} 
The authors of this paper present BurstTracker. BurstTracker can be used to detect if the LTE downlink
is the bottleneck for developer applications. BurstTracker identifies bursts within a prolonged transfer 
on the LTE downlink. Bursts are contiguous periods of transfer during which the user's queue at the base
station is nonempty. When the application has a burst the downlink is being bottled. BurstTracker enabled
the authors to find that three Tier-1 US providers appear to be operating transparent split-TCP middleboxes
that impact the performance of video streaming. Furthermore, they discover a simple workaround to 
disable this problem and observe a video-streaming bitrate improvement by up to 35 percent. The authors
tribute their motivation from a surprising observation they made while streaming video on a smartphone 
connected to an LTE base station. This lead them to follow up on their ideas and create BurstTracker. 
In the Background section we learn of throughput time, bursts, and categorizes a downlink bottleneck.
They then describe the challenges poised for creating BurstTracker, and follow up with how they solved
these challenges. Next, they provide an evaluation of BurstTracker, and embark on a case study relating 
to video streaming. Lastly, they go into details behind the video streaming and how Slow Start Restart 
is the cause of the slowness not the LTE downlink. They further state that SSR is being added by a TCP
middlebox. To fix this problem they send data from an unreserved port but further improve the idea by having
the video client perform frequent micro-fetches to keep the application from going idle.

\end{itemize}
\subsection{Third pass information}
\label{sec:third}
\begin{itemize}

\item {\it Strengths:} 
The paper has a good introduction. It's short but it clearly illustrates the problem, what they have done, and 
what they hope to show the reader of the paper. Furthermore, I also believe the background section was a plus
 because they clarify a few of the questions I had from reading the introduction. Another strength was that they 
 provided their code for use and provided two ways of using their code: online and offline. 
 
\item {\it Weaknesses:} 
I thought the way they handled their adverse conditions for the smartphone was a bit weird. They even go on to 
say that they covered the phone with a metal container. I feel that their are better ways to provide adverse 
conditions for testing. I also felt that their evaluation section was really short and lacked content to prove the 
validity of BurstTracker. They kinda just quit talking about BurstTracker and focus on the video streaming after
the first half.

\item {\it Questions:} 
They mention that they were able to infer the contention and underflow milliseconds using only the users
own resource allocation in a base station that gave the millisecond to one user. I wonder if there inference 
would still work for a base station that shares the millisecond. 

\item {\it Interesting citations:}
I was actually really curious about the video stream behind mobile networks that is why I choose to look into
piStream~\cite{pistream}. piStream is a physical layer informed adaptive video streaming that happens over LTE.

\item {\it Possible improvements:} 
I felt that their use case on SSR overtook the paper from BurstTracker. The user cases provided good 
evidence of the use and necessity of BurstTracker but in my opinion it soon became the center Idea 
for a paper that was originally on BurstTracker and its evaluation. 

\item {\it Future work:} 
I'm sure you can create similar scripts to infer other statistics about resource blocks. They also mention how 
this can be used for 5G. Furthermore, you can use BurstTracker to investigate and explain other parts
of networking that aren't normally open to everyday users like video streaming. 

\end{itemize}

{
  \small 
  \bibliographystyle{acm}
  \bibliography{biblio}
}
\end{document}




