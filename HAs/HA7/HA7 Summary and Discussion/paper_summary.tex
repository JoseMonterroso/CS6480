\documentclass[letterpaper,twocolumn,10pt]{article}
\usepackage{epsfig,xspace,url}
\usepackage{authblk}


\title{CS 6480: Paper reading summary\\
HA 7.a\\}
\author{José Monterroso}
\affil{School of Computing, University of Utah}

\begin{document}

\maketitle
\section{The Design Philosophy of the DARPA Internet Protocols}

Paper discussed in this summary is ``The Design Philosophy of the DARPA Internet Protocols'~\cite{openflow.wp}.

\subsection{First pass information}
\label{sec:first}

\begin{enumerate}

\item {\it Category:} 

This paper is a bit old, but it's a paper on the analysis of an existing system. The existing system being analyzed is the 
the internet design from DARPA from 1988. 

\item {\it Context:} 

The technical area of this paper relates to internet protocols. The paper mentions IP, TCP, and other ISO protocols. 
This paper does not relate to any papers we have read in the class. However, TCP, IP, and ISO protocols are some 
of the things I learned in undergraduate networks. 

\item {\it Assumptions:} 

The author makes the assumption that DARPA papers and specification of internet protocols are difficult to read
and therefore he has to make this paper to help people understand the early reasoning which shaped the internet. 
Honestly, I've never seen or read any DARPA papers so I can't validate the assumption but some of the references
he mentions are about 40 plus pages in length.

\item {\it Contributions:} 

This paper attempts to capture some of the early reasoning which shaped the Internet protocols. He does 
mention that DARPA papers and specifications are hard to read so he contributes by providing a summary 
of the 1988 internet and its protocols. 

\item {\it Clarity:} 

From what I have read from this paper it seems to be well written. The paper is easy to read and offers some great 
insights into the world of computer networking from 1988.

\end{enumerate}

\subsection{Second pass information}
\label{sec:second}

\begin{itemize}

\item {\it Summary:} 

This paper describes a view of the original objectives of the internet architecture. Furthermore, it 
discusses the goals of this view (for the internet) and explains some of the important features of the 
protocols. The main goal for DARPA at the time was to create and develop a great technique for 
interconnecting and thus utilizing existing networks. At the time the internet was described to be: a
packet switched communications facility in which a number of distinguishable networks were connected
together using packet communications processors called gateways. Later in the paper was discussed a few second level goals that solidify the meaning of
DARPA's internet. DARPA's internet was designed to operate in a military context, that is why goal
number one is that communications must continue despite loss of networks or gateways. We then
touch a little on the architecture and engineering of the internet, datagrams, and TCP.

\end{itemize}

\subsection{Third pass information}
\label{sec:third}
\begin{itemize}

\item {\it Strengths:} 

I thought the abstract and the introduction were really straightforward and did a great
job of introducing the problem at hand. I also think having the list of goals of the internet at the beginning before 
going into depth for each goal was a strength because it offered a nice overview rather than listing a bunch of
topics related to the internet. It's always nice to have the 'why' at the beginning. Lastly, I really like how the author
provided little examples for pretty much every little point he made. 

\item {\it Weaknesses:} 

The author mentioned a lot of troubles with people and networking. He mentions the problem and explains why 
they are problems but doesn't really offer any help. For example, the problem with implementing everything at the
endpoints. He describes the problem, but all he says is that people mess up and need to do better (i.e. problems 
happen all the time). Another example is when he discussed the simulator he simply states ``we do not have a 
good idea on how to offer guidance..." Although he does meet his goal of discussing the intended idea behind the internet
and offering background information, I do feel that he could have added more to his paper. Even though this paper is 
old, I feel that all he is doing is combining typical internet concepts and providing a broad summary. 

\item {\it Questions:} 

The paper mentions that the first internet was a connection of a variety of networks. I wonder how easy or hard it
was for them to connect these networks so that they can all operate with datagrams. 

\item {\it Interesting citations:} 

In undergraduate computer networks, I enjoyed learning about TCP. I remember studying the different 
state machines that related to each TCP version. So I find the TCP~\cite{tcp} RFC from 1981 to be an 
interesting read. 

\item {\it Possible improvements:} 

The author discusses how TCP went through a few versions before settling on the current one at the time. I would 
liked to have seen figures of these transformations and how each version improved and why. In the general sense
I do think tables of figures would have greatly improved this paper. 

\item {\it Future work:}

This paper offers a small introduction to the world of the 1988 internet. I think it would be interesting to do a paper on
how the internet has shifted to the way we do things now. The author also mentions a `simulator',  and how difficult it 
was to create at the time, I'm curious if this is still the case today.

\end{itemize}

{
  \small 
  \bibliographystyle{acm}
  \bibliography{biblio}
}
\end{document}


























