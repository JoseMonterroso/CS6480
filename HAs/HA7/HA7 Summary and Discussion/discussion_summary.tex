\documentclass[letterpaper,twocolumn,10pt]{article}
\usepackage{epsfig,xspace,url}
\usepackage{authblk}


\title{CS 6480: Class discussion summary\\
HA 7.b\\}
\author{José Monterroso}
\affil{School of Computing, University of Utah}

\begin{document}

\maketitle
\section*{Discussion summary}

\begin{itemize}

\item {\it Summary:} 
This was a very old paper from 1988, that at the time was published 15 years after the internet
was commercialized. The author makes a bold claim that information about the internet is generally 
available but not well understood. I can agree with this because even in his own paper he didn't 
really provide much detail regarding the implementation aspect of the internet. We than transitioned 
into the meat of the paper. We discussed how DARPA's top level goal was to interconnect networks 
together. Section three of the paper provides us with a list of goals which were established for the
internet architecture. If the seven goals were listed in any different order the internet would have 
been very different today. Lastly, during our discussion we noticed that security was not as important 
back then, and thus creates bigger problems for us today. This paper helped me realize that the 
process for the internet was not as crisp or elegant as I thought it was. 

\item {\it Strengths and weaknesses:} 
I can't really say for sure what the class consensus on this paper was. Not really much was said
about the strengths or weaknesses of this paper during our class discussion. But on a personal 
note, I can say that the age of this paper comes into play. For example, at the time of this paper's
release back in 1988, this paper was already roughly 15 years late from the development of the DARPA
net. This in turn begs the question what is the use of this paper. This paper helps readers understand a 
little about why the internet was chosen to be the way it was and why it grew to become the way it is. 
But that's about it. Even though this paper was a conference paper I felt 
it was lacking content, especially when it came to architecture and implementation as well as TCP. 

\item {\it Connection with other work:} 
There really isn't any papers from our previous discussions that I can directly relate this paper to 
because we have started a new section on internet architectures. However, what I can say is that 
a lot of the things that are mentioned in this paper are the reasons why NFV and cloud computing 
were implement the way they are today. 

\item {\it Future work:}
A few common things
were mentioned. For example, this week we transitioned to learning about networking architectures,
so it was brought up that we could look into or develop networking architectures. Ideally, these type
of networks would have an emphasis on security and reliability. It was also mentioned that we could 
find/create better ways to make the internet more accountable. Lastly, future work for the author himself
would be to expand on this paper and to dive a little deeper into the details of datagrams and TCP. 

\end{itemize}
\end{document}

\begin{comment}
*Transition now to network architecture 

*Very old paper, published in ACM SIGCOM conference 

*paper published 15 years after the internet was commercialized

*Claim that info about the internet is generally available but not  well understood 

*IETF => RFCs

*Top level goals for the DARPA was to interconnect networks together 

*The process for the internet was not as crisp or elegant as we think it was

*multi-media network = multi transport pipes/ multi transport network 
	=> Argue it might have been efficient, but not as flexible or robust 
	
*AT\&T had monopoly 
	=> can we build networks administered by other  companies
	
*The technique for multiplexing was packet switching 
	=> most networks were already packet switched 
	
*Nodes
	=> gateways
	=> endpoint hosts

*If the goals in section 3 were in different order the internet would have been very different today

Phone networks had large centralized centers 
	=> five or ten for the whole country 
	=> that is why we have this as the top goal in section 3
	
Which section 2 goals are the best for todays
	=> cost effective for todays commercialized internet
	=> need to add security to the list of goals
		=> easier to attack ppl rather than organizations

For most attacks in the internet the attackers want the internet to stay up
	=> no one can pay you back  
	
Motivation to stop the internet 
	=> state hackers
	
The threat was nuclear attack from nation states
	=> the threat today was potential taking down the internet 
	
Synchronization
	=> it was the assumption in this architecture that synchronization would never be lost 
	=> We can keep synchronization at the endpoints

Fate-sharing
	=> data only at the endpoints 
	=> synchronization is stored in the host 
	=> if you are gone then its okay for the state to be gone 
	
There is no check at an endpoint
	=> can send more traffic than an endpoint can handle
	
TCP didn't work for the debugger
	=> might impede the debugging info
	=> a debugger protocol should not be reliable 

Hard to have multiple services
	=> proved more difficult than first hoped if the underlying network can not explicitly support it 
	=> only so much you can do at the transort layer depending on the underlying tech 
	

You have an architecture but the architecture is lossy gooosy, but you ahve to implmement it in differentt ways
the impliementation is important but it's hard to do this becasue ttheir is not enough detail for them to be made correctly 


*Future work ideas
	=> Changinf of the internet architecute
	=>  possible or insentive to change it
	=> Their future work would be to develop open flow, and write more in detail things like tcp things, futute internet architectures 
	=> Better ways of internet accountability 
\end{comment}




























































