\documentclass[letterpaper,twocolumn,10pt]{article}
\usepackage{epsfig,xspace,url}
\usepackage{authblk}


\title{CS 6480: Paper reading summary\\
HA 9.a\\}
\author{José Monterroso}
\affil{School of Computing, University of Utah}

\begin{document}

\maketitle
\section{Title of paper: Making 5G a Reality}

Paper discussed in this summary is ``Making 5G a Reality''~\cite{5gwhite}.

\subsection{First pass information}
\label{sec:first}

\begin{enumerate}

\item {\it Category:} 
From doing the first pass we can explicitly see that it says that this paper is a white paper relating to 5G
technologies. Furthermore, this paper is an analysis of the Third Generation Partnership Project (3GPP) 
5G specifications. 

\item {\it Context:}
The technical area of this paper relates to 5G specifications and the impact on Business Support Systems 
(BSS) and Operational Support Systems (OSS). This paper relates to most of our previously discussed papers.
5G will require the use of cloud computing and NFV to reduce costs and enable softwarization and virtualization,
this makes the network more agile and flexible. 

\item {\it Assumptions:} 
I didn't see any major assumptions during my first pass. But the only thing that caught my eye was that they
mentioned that mobile communication has seen a tremendous growth and will continue to see a major 
growth. I believe this assumptions is valid because we have seen this trend ever since generation 1. Furthermore,
the growth of wireless devices keeps growing as well. 

\item {\it Contributions:} 
The goal of this white paper is to bring information regarding 5G from 3GPP standardization perspective and 
implications on BSS/OSS. It's paper contributes by bringing 5G information and implications to CEOs and 5G
product developers. 

\item {\it Clarity:} 
From what I have read this paper appears to be written well. I looks to have lots of figures and
diagrams. However, this paper does contain a lot of industry jargon. 

\end{enumerate}

\subsection{Second pass information}
\label{sec:second}

\begin{itemize}

\item {\it Summary:} 
It appears that 5G will bring huge changes to the industry and society. There are two main reasons for the
increasing demand of services by the mobile industry. The first being due to digitization or mobile operator
partnerships with the media industry and over the top service providers. A second being the increase in 
data-rate of mobile communication systems and the need for the mobile industry to grow and diversify. 
Internet of Things (IoT) devices will be an integral part of 5G. Subscriber identification and authentication 
is achieved today by SIM cards. With the growing number of IoT devices we need simpler methods like
softwarization of the SIM card. 5G will come in 3 different phases, NSA refers to 5G radio using the 4G 
network core. Phase 1 will cover the radio, core, security and all associated specifications. Finally,
Phase 2 will include the rest of the technology specifications for massive machine-type communication.
The new radio is expected to support much higher data-rates with lower latency. NG-RAN is the next
generation RAN, it is faster and has lower latency than its predecessor. We then touch a lot into all the 
different components of the Control plane and User plane. Next they discuss a little bit on networking 
slicing and uses for security and network isolation. Lastly, we slightly touch on service based architecture
before moving on to 5G security. Security in 5G systems relies on primary and secondary authentication
as well as additional security for NFV, open source, IP based interfaces, etc...

\end{itemize}

\subsection{Third pass information}
\label{sec:third}
\begin{itemize}

\item {\it Strengths:} 
The first half of the paper does a great of introducing 5G, and some of the necessary functionalities that 
go with it. The authors of the white paper do a great job of tying in multiple networking concepts. For 
example in a service based architecture they bring up the use of NFs and cloud based technologies, 
both of which we have studied in past lecture papers. The figures in the paper do a significant job 
of allowing me to visualize the 5G space. Furthermore, I like how the authors provide little examples
of the concepts they are discussing. 

\item {\it Weaknesses:} 
I'm not at all familiar with 4G networking technologies. So when the authors of this paper default 
to explaining a new 5G concept as a twist on an older 4G idea, I have no idea what they are referring 
to. It seemed a bit common in the security section as well as explaining some of the architectural points of 
5G to default to 4G ideas. A lot of jargon was used which was difficult for me. Also there are so many acronyms used to describe
5G technologies that I simply couldn't memorize nor keep up. 

\item {\it Questions:} 
I just have a lot of questions that relate to 4G technologies, simply because the 5G technology is an extension 
of what was already used. Furthermore, I question their general use of open source technology. A lot of open source
tech is great because it's open to everyone, but it can have security threats and may be difficult to use. 

\item {\it Interesting citations:} 
I found 5G security to be an interesting read. Especially when they mentioned multiple 3GPP standards 
and previously used security methods. That is why I am choosing to look over the 3GPP security guidelines 
~\cite{3gppsecurity}.

\item {\it Possible improvements:}
I was gonna mention a list of abbreviations but I didn't notice they already had one until later in to the paper. I think what 
would also help is if they simply explain everything without reference or using 4G equivalent technology in their
definitions.

\item {\it Future work:} 
They mention how 5G has to be at certain data speeds and latency standards so I can imagine various types
of measurement papers that could be created. POWDER's capabilities lie within the realm of 5G technologies. 
In the REU I never got a chance to play around with 5G tech (eNodeB) so I can see myself using these type of
things in the future.

\end{itemize}

{
  \small 
  \bibliographystyle{acm}
  \bibliography{biblio}
}
\end{document}
































