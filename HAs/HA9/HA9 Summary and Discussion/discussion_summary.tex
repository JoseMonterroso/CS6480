\documentclass[letterpaper,twocolumn,10pt]{article}
\usepackage{epsfig,xspace,url}
\usepackage{authblk}


\title{CS 6480: Class discussion summary\\
HA 9.b\\}
\author{Your Name}
\affil{School of Computing, University of Utah}

\begin{document}

\maketitle
\section*{Discussion summary}

\begin{itemize}

\item {\it Summary:} Summarize the class discussion of the papers. Note in
particular anything that came out of the discussion that you missed in your
reading, or that made you change your mind. 

\item {\it Strengths and weaknesses:} Note the consensus
in the class, or your own new insights, regarding the strengths and
weaknesses of the paper(s).

\item {\it Connection with other work:} Describe any connections that were made in
the class discussion to other papers (with citations). (Or connections that you realized
as a result of the class discussion.)

\item {\it Future work:} Briefly describe any possibilities
for future work that came out of the discussion and/or were triggered
in your mind because of the discussion.

\end{itemize}
{
  \small 
  \bibliographystyle{acm}
  \bibliography{biblio}
}

\end{document}

\begin{comment}

the white paper is ta technical write up, not published anywhere 
	=> not peer reviewed 

Goal is to summarize 3GPP and add NEC opinion on 5G


IoT devices are overwhelming majority don't send a lot of data

Open-ended questions (lots of business versus technical)
	=> market for 5G?
	=> is our tech ready for 5G adoption 

classic things providers worry about
	=> data rates, services provisioned by the mobile industry 
	
verticals - ISP need to partner with other companies 
	=> 3rd party wants to tinker with the network 
		=> wants to optimize the network for their work 
	=> can we apply machine learning to optimize these 
	
They talk about SDN, Cloud, NFV (virtualization)
	=> these are the things we have been looking at 
	
5G motivation how to make money and make a business run

The rise of Open source!
	=> lots of use in it 
	
UE - user equipement 
	- ex smart phone 
	
RAN - constituting the UE and RAN tech 
 
 The core network is fairly centralized 
 	=> when 4g was fairly well deployed yoou woould have 10 20 centralized locations where most of the core network
	elements would be resident to cover all the united states
		=> lots of base statation that funnel back to each core network data centers 
		=> 5G doesnt do this becasue they want higher data rate and lower latency 
		
CP vs UP
	=> CP blue, UP red 
	
AMF - control mobiliity within the core network 
	=> you have to authenticate to get in

UDM - user subscriber data is saved
	=> 
	
AUSF - Authenticate he user 

Modular architecture 
	=> different authentication policies possible 

Everything in a mobile network is stateful 

SMF - is managing the sessions state 
	=> 

Actually IP network underneath the 5G components
	=> the 5G components communicate over that IP network 
	
The tunnel moves with you 
	=> keep your IP address
	
PCF- policty function

AF - third parties to influence the policy  
	=> expose API to the network 
	
5G SERVICES
	=> eMBB - mobile hotspots 
	=> MMTC - IoT communication (lots and lots of devices )
	=> low latency communication - there might be application that really need low level latency 
	
5G and IoT 
	=> in 5G this is a big category 
	=> in 4G it was added afterwards 
	
next generation eNodeB  - 5g network 

gNB - split into Central unit and Distributed Unit 
	=> DU deals with RF environment 
	=> you have the F1 interface for flexibility 
	
	
RAN functionality 
	=> Look at slides 
	
	
\end{comment}


























































